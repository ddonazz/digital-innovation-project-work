\begin{abstract}
    Il presente lavoro di progetto universitario si concentra sull'analisi approfondita e sull'implementazione pratica di un Gateway API utilizzando Spring Cloud Gateway. \\
    In un'era di trasformazione digitale rapida, la gestione strategica delle API è emersa come un fattore critico per le aziende che adottano architetture a microservizi, dove la complessità della comunicazione inter-servizio, della sicurezza e del routing può diventare proibitiva senza un punto di ingresso centralizzato. \\
    Questo studio si propone di esplorare Spring Cloud Gateway come soluzione robusta e reattiva per affrontare tali sfide.  

    La metodologia adottata per questo progetto combina elementi di indagine ("Survey"), elaborazione ("Elaboration") e applicazione pratica ("Application"), come delineato nelle linee guida del Project Work. \\
    È stata condotta un'indagine sui concetti fondamentali dei Gateway API e sulle loro funzionalità comuni, seguita da un'elaborazione dettagliata dell'architettura e delle caratteristiche specifiche di Spring Cloud Gateway. \\
    Il cuore del progetto consiste nell'implementazione di un prototipo di Gateway API, disponibile nel repository Git fornito (https://github.com/ddonazz/api-gateway), che dimostra funzionalità chiave quali il routing dinamico, la gestione centralizzata della sicurezza (autenticazione e autorizzazione), la limitazione del tasso di richieste (rate limiting) e la gestione delle eccezioni.  

    I risultati ottenuti evidenziano i significativi vantaggi di Spring Cloud Gateway, tra cui una maggiore sicurezza, un'accelerazione dell'innovazione e tempi di commercializzazione ridotti, una migliore esperienza utente e una conformità normativa più efficiente. \\
    Tuttavia, l'analisi critica ha anche rivelato sfide intrinseche all'adozione di un Gateway API, come l'aumento della complessità architetturale e il potenziale di introduzione di latenza aggiuntiva. \\
    Il lavoro si conclude con una discussione sui contributi del progetto e sulle prospettive future per superare le limitazioni identificate, proponendo sviluppi che potrebbero ulteriormente migliorare la robustezza e l'applicabilità di tali soluzioni.
\end{abstract}