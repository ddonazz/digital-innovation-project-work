Questa sezione descrive l'implementazione pratica di un Gateway API utilizzando Spring Cloud Gateway, come dimostrato nel repository Git fornito (\url{https://github.com/ddonazz/api-gateway}). \\
L'obiettivo è illustrare come i concetti teorici discussi nei capitoli precedenti siano stati tradotti in una soluzione funzionante, evidenziando le scelte di design e i dettagli implementativi.

\section{Architettura e Design del Gateway Implementato}

L'architettura del sistema implementato è composta da un servizio Spring Cloud Gateway che funge da punto di ingresso centrale e da uno o più semplici microservizi backend. \\
Per scopi dimostrativi, è stato configurato un servizio di backend minimale (ad esempio, un \texttt{product-service} o un \texttt{user-service} come si vede in progetti simili), che espone alcune API REST.

Il flusso delle richieste è il seguente:
\begin{itemize}
    \item Un client (ad esempio, un browser o un'applicazione mobile) invia una richiesta HTTP al Gateway API.
    \item Il Gateway API, basato su Spring Cloud Gateway, intercetta la richiesta.
    \item Utilizzando le route e i predicati configurati, il Gateway determina il microservizio backend appropriato a cui inoltrare la richiesta.
    \item I filtri configurati (globali o specifici per la route) vengono applicati per modificare la richiesta in ingresso o la risposta in uscita.
    \item La richiesta viene inoltrata al microservizio backend.
    \item Il microservizio elabora la richiesta e restituisce una risposta al Gateway.
    \item Il Gateway applica eventuali filtri post-elaborazione alla risposta prima di inviarla al client.
\end{itemize}

Le scelte di design hanno privilegiato una configurazione dichiarativa delle route e dei filtri tramite file \texttt{application.yml}, sebbene Spring Cloud Gateway supporti anche la configurazione programmatica in Java. \\
Per la scoperta dei servizi è stata utilizzato un server Eureka. L'ambiente di sviluppo è stato avviato utilizzando Spring Initializr, che facilita la creazione di progetti Spring Boot con le dipendenze necessarie, supportando sia Gradle che Maven.

Una chiara rappresentazione dell'architettura e delle scelte di design dimostra non solo la capacità tecnica, ma anche una profonda comprensione dei principi architetturali, collegando efficacemente la teoria (Capitoli 2 e 3) con la pratica. \\
La selezione di servizi backend semplici ma rappresentativi consente di illustrare in modo efficace le capacità di routing e filtraggio del gateway.

\section{Dettagli Implementativi e Configurazione}

Questa sezione fornisce dettagli specifici sull'implementazione delle funzionalità chiave all'interno del progetto API Gateway, con esempi di configurazione e codice.

\section{Definizione delle Route e dei Predicati}

\subsection[EUREKA]{Eureka e la Scoperta dei Servizi}
La scoperta dei servizi su un server \textbf{Eureka} semplifica l'individuazione e l'interazione tra i microservizi.
Invece di configurare manualmente gli indirizzi di rete per ogni servizio, i servizi si registrano con Eureka, che mantiene un registro aggiornato delle istanze di servizio disponibili.
Quando un client necessita di comunicare con un servizio specifico, interroga Eureka per ottenere la posizione attuale di un'istanza sana di quel servizio.

Le route, come quelle definite nel file \texttt{application.yml} per chiarezza e facilità di configurazione, sfruttano questa capacità di scoperta.

Ogni route specifica un \textbf{ID} univoco, l'\textbf{URI} del servizio di destinazione (spesso un nome logico registrato con Eureka, ad esempio \texttt{lb://NOME-DEL-SERVIZIO}), e una serie di \textbf{predicati} per la corrispondenza delle richieste in entrata.
Questi predicati determinano se una richiesta deve essere instradata a un particolare servizio. Ad esempio, una route potrebbe specificare che tutte le richieste che iniziano con \texttt{/api/utenti} devono essere indirizzate al servizio \texttt{lb://SERVIZIO-UTENTI}.
Questo approccio disaccoppia i client dalle posizioni fisiche dei servizi, migliorando la resilienza e la manutenibilità dell'architettura a microservizi.

Un esempio di configurazione di una route in YAML è il seguente:

\lstinputlisting[style=YAMLStyle, caption=Configurazione delle rotte]{code/routes-configuration.yml}

In questo esempio, la route \texttt{user\_service\_route} inoltra le richieste GET e POST con percorso \texttt{/api/users/**} al servizio utente, rimuovendo il prefisso \texttt{/api}. \\
La route \texttt{product\_service\_route} inoltra le richieste con percorso \texttt{/api/products/**} e un header \texttt{X-Version} con valore \texttt{v2} al servizio prodotto, aggiungendo un header \texttt{X-Forwarded-By}. \\
L'uso di predicati come Path, Method e Header dimostra la flessibilità nella definizione delle regole di routing.

\section{Implementazione dei Filtri Personalizzati}
Sono stati implementati filtri personalizzati per dimostrare la capacità di estensione di Spring Cloud Gateway oltre i filtri predefiniti.

\lstinputlisting[style=JavaStyle, caption=Esempio di GatewayFilter per la validazione di un header di autorizzazione]{code/AuthorizationHeaderFilterFactory.java}
Questo filtro personalizzato (simile all'esempio di validazione precedente) verifica la presenza dell'header \texttt{Authorization} e, in caso di assenza, genera un errore 401 (Unauthorized).

\subsection{Sicurezza Delegata nel Microservizio}

In un'architettura a microservizi, la gestione della sicurezza è spesso centralizzata a livello di API Gateway. 
Il Gateway ha il compito di autenticare le richieste provenienti dai client esterni (ad esempio, validando un token JWT) e, una volta verificata l'identità dell'utente, arricchisce la richiesta inoltrata ai servizi interni con le informazioni necessarie per l'autorizzazione.

Questo capitolo descrive l'implementazione di un filtro di sicurezza all'interno di un singolo microservizio, sviluppato con Spring WebFlux, che opera secondo questo principio di \textit{autenticazione delegata}.
Il servizio non esegue una nuova autenticazione, ma si fida delle informazioni (come utente e ruoli) inserite negli header HTTP dal Gateway.

\subsubsection{Configurazione della Catena di Filtri di Sicurezza}

Il cuore della configurazione di sicurezza risiede nella classe \texttt{SecurityConfig}.
Grazie alle annotazioni \texttt{@EnableWebFluxSecurity} e \texttt{@EnableReactiveMethodSecurity}, attiviamo il supporto di Spring Security per applicazioni reattive e per la sicurezza a livello di metodo.

Il bean \texttt{SecurityWebFilterChain} definisce la catena di filtri che ogni richiesta deve attraversare.

\lstinputlisting[style=JavaStyle, caption={Definizione della catena di filtri di sicurezza in \texttt{SecurityConfig.java}.}, label={lst:securityconfig}]{code/product-service/SecurityConfig.java}

I punti salienti di questa configurazione sono:
\begin{itemize}
    \item \textbf{Disabilitazione dei meccanismi standard}: Le funzionalità di login basate su form (\texttt{formLogin}), autenticazione HTTP Basic (\texttt{httpBasic}), protezione CSRF (\texttt{csrf}) e logout sono disabilitate. Questo perché il servizio non gestisce sessioni utente tradizionali, ma opera in modo \textit{stateless}, processando ogni richiesta in base alle informazioni che essa contiene.
    \item \textbf{Autorizzazione predefinita}: Con \texttt{anyExchange().authenticated()}, si stabilisce che ogni richiesta, a qualsiasi endpoint, deve essere autenticata per poter essere processata.
    \item \textbf{Filtro personalizzato}: Viene creato e inserito un \texttt{AuthenticationWebFilter}. Questo filtro è il componente chiave che orchestra il processo di estrazione dell'identità. Viene posizionato esattamente al punto della catena in cui Spring Security si aspetta che avvenga l'autenticazione (\texttt{SecurityWebFiltersOrder.AUTHENTICATION}).
\end{itemize}

\subsection{Estrazione dell'Identità dagli Header HTTP}
Il compito di leggere gli header HTTP e tradurli in un oggetto \texttt{Authentication} comprensibile a Spring Security è delegato alla classe \texttt{HeaderAuthenticationConverter}.

\lstinputlisting[style=JavaStyle, caption={Implementazione di \texttt{HeaderAuthenticationConverter.java}.}, label={lst:converter}]{code/product-service/HeaderAuthenticationConverter.java}

Questo convertitore implementa una logica semplice ma efficace:
\begin{enumerate}
    \item Cerca due header specifici nella richiesta in arrivo: \texttt{X-Auth-User} per l'identificativo dell'utente e \texttt{X-Auth-Roles} per i suoi ruoli, separati da virgola.
    \item Se gli header non sono presenti, restituisce un \texttt{Mono.empty()}, segnalando che non è stato possibile costruire un'identità per la richiesta corrente.
    \item Se gli header sono presenti, costruisce un oggetto \texttt{UsernamePasswordAuthenticationToken}, che è un'implementazione standard di \texttt{Authentication}. Questo oggetto contiene il principal (l'utente), le credenziali (impostate a \texttt{null} perché non pertinenti in questo contesto) e le autorità (i ruoli).
\end{enumerate}

È fondamentale notare che questo componente si fida implicitamente del fatto che gli header siano stati inseriti da un'entità affidabile (l'API Gateway) e che non siano stati manomessi.

\subsection{Autorizzazione Basata su Ruoli}
Una volta che il \texttt{SecurityContext} è stato popolato con l'oggetto \texttt{Authentication}, il microservizio può implementare logiche di autorizzazione granulari a livello di singolo endpoint. Questo viene realizzato tramite l'annotazione \texttt{@PreAuthorize}, che sfrutta Spring Expression Language (SpEL) per definire le regole di accesso.

\lstinputlisting[style=JavaStyle, caption={Esempi di autorizzazione nel \texttt{ProductController.java}.}, label={lst:controller}]{code/product-service/ProductController.java}

Come mostrato nel \texttt{ProductController} (listato \ref{lst:controller}):
\begin{itemize}
    \item L'endpoint \texttt{/admin-dashboard} è accessibile solo agli utenti che possiedono il ruolo \texttt{'ADMIN'}.
    \item L'endpoint \texttt{/user-profile} è accessibile sia agli utenti con ruolo \texttt{'USER'} che \texttt{'ADMIN'}.
    \item È inoltre possibile accedere programmaticamente all'identità dell'utente autenticato iniettando un \texttt{Mono<Principal>} nel metodo del controller, come dimostra l'endpoint \texttt{/authenticated-user-info}.
\end{itemize}

Questo approccio crea un sistema di sicurezza efficiente e disaccoppiato, dove il microservizio delega l'onere dell'autenticazione primaria e si concentra esclusivamente sulla logica di autorizzazione basata sulle informazioni affidabili fornite dal Gateway. // 

\section{Altre Funzionalità Implementate}
\begin{itemize}
    \item \textbf{Limitazione del Tasso (Rate Limiting):} È stata configurata la limitazione del tasso per alcune route, utilizzando il filtro \texttt{RequestRateLimiter} integrato di Spring Cloud Gateway. Questo filtro consente di definire quante richieste un utente (identificato per IP attraverso la condifuirazione di un bean \textit{ipKeyResolver}) può effettuare in un dato periodo di tempo.
    \lstinputlisting[language=Yaml, style=YAMLStyle, caption=Configurazione YAML per Rate Limiting]{code/rate-limit.yml}
    \item \textbf{Integrazione Circuit Breaker:} Sebbene non esplicitamente implementato nel repository fornito dall'utente, un'integrazione con un circuit breaker (es. Resilience4j, successore di Hystrix) sarebbe configurata per proteggere le route da servizi backend lenti o non disponibili.
    \item \textbf{Logging e Monitoraggio:} La configurazione di logging standard di Spring Boot è stata utilizzata per monitorare il traffico e gli errori. Per un sistema di produzione, si integrerebbe con soluzioni esterne come Splunk o ELK Stack per un monitoraggio più avanzato.
\end{itemize}

\begin{table}[htbp]
\small
\centering
\caption{Funzionalità Implementate nel Progetto API Gateway}
\label{tab:funzionalita_implementate}
\begin{adjustbox}{max width=\linewidth, center} 
\begin{tabularx}{1.2\linewidth}{
    >{\RaggedRight\arraybackslash}p{0.22\linewidth}
    >{\RaggedRight\arraybackslash}X
    >{\RaggedRight\arraybackslash}X
    >{\RaggedRight\arraybackslash}X
}
\toprule
\textbf{Funzionalità} & \textbf{Descrizione dell'Implementazione} & \textbf{Riferimento Codice/Configurazione} & \textbf{Endpoint di Test/Esempio} \\
\midrule
Routing Basico & Route configurate per inoltrare richieste a servizi utente e prodotto. & \texttt{application.yml} (route \texttt{user\_service\_route}, \texttt{product\_service\_route}) & \lstinline|GET /api/users/1|, \lstinline|GET /api/products/abc| \\
\midrule
Predicati Multipli & Uso combinato di Path e Method, e Path e Header. & \texttt{application.yml} (route \texttt{user\_service\_route}, \texttt{product\_service\_route}) & \lstinline|GET /api/users/1|, \lstinline|GET /api/products/abc| con \texttt{X-Version: v2} \\
\midrule
Filtro Personalizzato (Logging) & GlobalFilter per registrare il tempo di elaborazione di ogni richiesta. & \texttt{RequestTimeLoggingFilter} & Tutte le richieste al gateway (output console) \\
\midrule
Filtro Personalizzato (Validazione Header) & GatewayFilter per verificare la presenza dell'header Authorization. & \texttt{AuthorizationHeaderFilterFactory} & \lstinline|GET /api/protected/resource| (senza header Authorization $\rightarrow$ 401) \\
\midrule
Limitazione del Tasso & Configurazione del filtro RequestRateLimiter per una route pubblica. & \texttt{application.yml} (route \texttt{public\_api\_rate\_limited}) & Richieste multiple a \lstinline|GET /api/public/data| (superando il limite $\rightarrow$ 429 Too Many Requests) \\
\bottomrule
\end{tabularx}
\end{adjustbox}
\end{table}

La dettagliata implementazione di funzionalità specifiche, come i filtri personalizzati e la gestione della sicurezza, dimostra l'estensibilità e l'adattabilità di Spring Cloud Gateway oltre le sue capacità predefinite. Ciò evidenzia un livello più profondo di padronanza, andando oltre una semplice indagine (\enquote{Survey}) per abbracciare pienamente gli aspetti di applicazione ed elaborazione (\enquote{Application} ed \enquote{Elaboration}) del progetto. Gli esempi specifici di configurazione e codice fungono da prova concreta del lavoro svolto e della comprensione dello studente.

\section{Esempi di Utilizzo e Dimostrazione Pratica}

Per validare il corretto funzionamento dell'API Gateway e l'integrazione con i microservizi sottostanti, è stato predisposto uno script di test automatizzato. Questo script, basato su comandi \texttt{curl} e \texttt{jq}, simula le interazioni di un client, verificando sistematicamente le diverse funzionalità implementate, come l'autenticazione, l'autorizzazione basata su ruoli, il rate limiting e i predicati di routing.

Si assume che il Gateway sia in ascolto sulla porta \texttt{8080} e che tutti i microservizi (autenticazione, prodotti, utenti) siano attivi e raggiungibili dal Gateway.

\begin{itemize}
    \item \textbf{Esecuzione dello script di test:} Per avviare la dimostrazione, è sufficiente eseguire lo script Bash, che si occuperà di orchestrare tutte le chiamate API e di riportare l'esito di ogni test.
    \item \textbf{Sezione 1: Servizio di Autenticazione e Generazione Token}
    \begin{itemize}
        \item \textit{Richiesta:} Ottenere i token JWT per un utente standard (\texttt{user}) e un amministratore (\texttt{admin}).
        \item \textit{Logica di Test:} Lo script invia due richieste \texttt{POST} all'endpoint \texttt{/auth/login} con le rispettive credenziali. I token restituiti vengono salvati per i test successivi. Viene inoltre testato un tentativo di login con password errata.
        \item \textit{Risultato Atteso:} I primi due login hanno successo (stato HTTP 200) e forniscono un token valido. Il login con credenziali errate fallisce, restituendo un errore (nello script, si attende un 500 Internal Server Error, gestito dal servizio di autenticazione).
    \end{itemize}
    
    \item \textbf{Sezione 2: Dimostrazione del Rate Limiter}
    \begin{itemize}
        \item \textit{Richiesta:} Sovraccaricare un endpoint per attivare il limite di richieste.
        \item \textit{Logica di Test:} Lo script esegue un ciclo che invia 15 richieste consecutive e in rapida successione all'endpoint \texttt{/auth/login}.
        \item \textit{Risultato Atteso:} Le prime richieste (fino al limite configurato, ad esempio 10 al secondo) ricevono uno stato HTTP 200. Superata la soglia, il Gateway risponde con uno stato HTTP 429 Too Many Requests, bloccando le richieste in eccesso senza inoltrarle al servizio sottostante.
    \end{itemize}

    \item \textbf{Sezione 3: Verifica dell'Accesso Protetto al Product Service}
    \begin{itemize}
        \item \textit{Richiesta:} Accedere a endpoint con diverse regole di autorizzazione.
        \item \textit{Logica di Test:} Vengono eseguiti molteplici test:
            \begin{enumerate}
                \item Accesso senza token: fallisce con stato 401 Unauthorized.
                \item Accesso con token valido ma senza l'header \texttt{X-Version: v2} richiesto dalla rotta: il Gateway non trova una rotta corrispondente e il test fallisce.
                \item Utente \texttt{user} accede a una risorsa per utenti (\texttt{/api/products/user-profile}): successo (200 OK).
                \item Utente \texttt{user} tenta di accedere a una risorsa per amministratori (\texttt{/api/products/admin-dashboard}): accesso negato (403 Forbidden).
                \item Utente \texttt{admin} accede sia alla risorsa per amministratori che a quella per utenti: successo in entrambi i casi (200 OK).
            \end{enumerate}
        \item \textit{Risultato Atteso:} I test dimostrano che il Gateway valida correttamente la presenza e la validità del token JWT, applica i predicati sugli header (\texttt{X-Version}) e inoltra gli header di autenticazione (\texttt{X-Auth-User}, \texttt{X-Auth-Roles}) al servizio \texttt{product-service}, che a sua volta applica l'autorizzazione basata sui ruoli.
    \end{itemize}

    \item \textbf{Sezione 4: Verifica dei Predicati di Metodo sul User Service}
    \begin{itemize}
        \item \textit{Richiesta:} Accedere a un endpoint con un metodo HTTP consentito e uno non consentito.
        \item \textit{Logica di Test:} La rotta per \texttt{/api/users/**} è configurata per accettare solo richieste \texttt{GET}. Lo script tenta prima una richiesta \texttt{GET} e poi una \texttt{DELETE}.
        \item \textit{Risultato Atteso:} La richiesta \texttt{GET} viene inoltrata correttamente, risultando in uno stato 200 (se la risorsa esiste) o 404 (se non esiste). La richiesta \texttt{DELETE}, invece, non corrisponde a nessuna rotta configurata e viene respinta dal Gateway con uno stato 404 Not Found, senza mai raggiungere il microservizio.
    \end{itemize}
\end{itemize}

Questi test pratici, orchestrati dallo script, confermano che l'API Gateway implementato agisce come un punto di controllo centralizzato, gestendo in modo efficace la sicurezza, il routing avanzato e la protezione dei servizi backend, soddisfacendo così i requisiti del progetto.