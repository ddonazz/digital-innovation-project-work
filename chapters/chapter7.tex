\section{Riepilogo dei Risultati}

Il presente Project Work ha affrontato lo studio e l'implementazione di un \emph{Gateway API} con \texttt{Spring Cloud Gateway}, un componente cruciale nelle moderne architetture a microservizi. Gli obiettivi iniziali del progetto, che includevano l'indagine sui concetti fondamentali dei Gateway API, l'elaborazione sulle caratteristiche di \texttt{Spring Cloud Gateway}, lo sviluppo di un'applicazione pratica e un'analisi critica della tecnologia, sono stati pienamente raggiunti.

Attraverso l'indagine, è stata consolidata la comprensione del ruolo essenziale di un Gateway API come \emph{punto di ingresso unificato}, capace di semplificare l'integrazione dei client e di centralizzare preoccupazioni trasversali come la \emph{sicurezza}, il \emph{rate limiting} e la \emph{resilienza}. L'analisi di \texttt{Spring Cloud Gateway} ha evidenziato la sua \emph{robustezza}, derivante dalla sua fondazione su un'architettura \emph{reattiva e non bloccante} (\texttt{Spring WebFlux} e \texttt{Project Reactor}), che lo rende particolarmente performante in scenari ad alta concorrenza. La \emph{modularità} intrinseca di \textsc{SCG}, basata su route, predicati e filtri, è stata riconosciuta come un fattore chiave per la sua flessibilità ed estensibilità.

L'implementazione pratica ha dimostrato con successo le capacità di \texttt{Spring Cloud Gateway} nel \emph{routing dinamico}, nella gestione centralizzata dell'\emph{autenticazione e dell'autorizzazione} (anche tramite \texttt{OAuth2}), nell'applicazione di \emph{filtri personalizzati} per la manipolazione delle richieste/risposte e nella \emph{limitazione del tasso di richieste}. Questi esempi concreti hanno convalidato la comprensione teorica e l'applicabilità della tecnologia.

L'analisi critica ha fornito una prospettiva equilibrata, riconoscendo i numerosi vantaggi di \textsc{SCG} in termini di sicurezza, accelerazione dello sviluppo e miglioramento delle prestazioni, ma anche evidenziando le sfide. Tra queste, l'introduzione di \emph{complessità architetturale}, il potenziale di \emph{latenza aggiuntiva} e le sfide legate alla gestione dello stato delle \emph{connessioni WebSocket} in scenari specifici. Questa valutazione ponderata è fondamentale per una comprensione completa della tecnologia.

\section{Contributi e Prospettive Future}

Il principale contributo di questo Project Work risiede nella combinazione di una rigorosa \emph{analisi teorica} con un'\emph{implementazione pratica tangibile} di \texttt{Spring Cloud Gateway}. Il progetto fornisce una risorsa completa per la comprensione di questa tecnologia, fungendo da guida per futuri sviluppi o per l'adozione in contesti reali. La documentazione dei dettagli implementativi e degli esempi di utilizzo pratico offre un punto di riferimento prezioso per chiunque intenda esplorare o implementare \textsc{SCG}.

Guardando al futuro, diverse aree di sviluppo potrebbero ulteriormente migliorare il progetto e la comprensione di \texttt{Spring Cloud Gateway}:

\begin{itemize}
    \item \textbf{Resilienza Avanzata:} Approfondire l'implementazione di pattern di resilienza come \emph{Circuit Breaker} (utilizzando \texttt{Resilience4j}) e \emph{Retry}, configurando fallback complessi per gestire i guasti dei servizi backend in modo più robusto.
    \item \textbf{Monitoraggio e Tracciamento Distribuito:} Integrare il Gateway con sistemi di monitoraggio e tracciamento distribuiti (es. \texttt{Zipkin} o \texttt{OpenTelemetry}) per ottenere una visibilità end-to-end sulle richieste attraverso l'intera catena di microservizi, facilitando il debugging e l'ottimizzazione delle prestazioni.
    \item \textbf{Gestione Avanzata di WebSocket:} Esplorare soluzioni e pattern architetturali per superare le limitazioni nella gestione dello stato delle connessioni \texttt{WebSocket} e nella capacità di broadcasting, rendendo il gateway più adatto per applicazioni real-time su larga scala.
    \item \textbf{Automazione del Deployment (\textsc{CI/CD}):} Sviluppare una pipeline \textsc{CI/CD} per automatizzare il building, il testing e il deployment del Gateway API e dei suoi servizi backend, migliorando l'efficienza operativa.
    \item \textbf{Test di Performance e Carico:} Eseguire test di performance e carico per valutare la scalabilità del Gateway in diverse condizioni di traffico e identificare potenziali colli di bottiglia.
\end{itemize}

Questi sviluppi futuri non solo affronterebbero le limitazioni identificate, ma spingerebbero anche l'applicazione di \texttt{Spring Cloud Gateway} verso scenari più complessi e di produzione, dimostrando una comprensione continua e un approccio proattivo alla risoluzione dei problemi. Il lavoro svolto in questo progetto serve da solida base per ulteriori ricerche e applicazioni pratiche nel campo dei Gateway API e delle architetture a microservizi.