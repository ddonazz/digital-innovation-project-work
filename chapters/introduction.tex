\section{Contesto e Motivazioni}

Il panorama dello sviluppo software contemporaneo è profondamente influenzato dall'adozione sempre più diffusa delle architetture a microservizi. Questo modello, che scompone le applicazioni monolitiche in un insieme di servizi piccoli, autonomi e accoppiati in modo lasco, offre numerosi vantaggi in termini di scalabilità, resilienza e agilità nello sviluppo. Tuttavia, l'adozione dei microservizi introduce anche nuove sfide significative, in particolare per quanto riguarda la gestione della comunicazione tra i servizi, la sicurezza, il monitoraggio e il routing delle richieste. Man mano che il numero di microservizi aumenta, la complessità di gestire le interazioni dirette tra client e servizi backend può diventare insostenibile, portando a problemi come la logica client frammentata e la difficoltà nella gestione delle preoccupazioni trasversali.

In questo contesto, il Gateway API emerge come un componente architetturale cruciale, fungendo da \textit{edge service} e da \textit{reverse proxy} che centralizza l'esposizione delle API e gestisce le preoccupazioni trasversali. Questo approccio risolve il problema \textit{N+1}, dove i client dovrebbero altrimenti effettuare chiamate separate a N servizi backend, aggregando le richieste e semplificando l'interazione. La gestione strategica delle API è, infatti, un fattore critico per le aziende che perseguono la trasformazione digitale, distinguendo i leader di mercato dai ritardatari.

La scelta di Spring Cloud Gateway (SCG) come tecnologia centrale per questo studio è giustificata dalla sua rilevanza all'interno dell'ecosistema Spring, ampiamente adottato nello sviluppo di applicazioni aziendali. SCG è una soluzione leggera e reattiva, costruita su Spring WebFlux e Project Reactor, che la rende particolarmente adatta per carichi di lavoro ad alta produttività e bassa latenza. Il presente Project Work si focalizza sullo studio di questa \textit{nuova tecnologia}, come richiesto dalle linee guida, fornendo dettagli implementativi e un'analisi critica della sua applicazione pratica. La necessità di un punto di ingresso centralizzato per le API è una conseguenza diretta dell'adozione dei microservizi; senza un Gateway API, la gestione delle interazioni tra client e servizi backend, la sicurezza e il monitoraggio diventerebbero estremamente complessi e frammentati, ostacolando l'efficienza e la scalabilità del sistema distribuito.

\section{Obiettivi del Progetto}

L'obiettivo primario di questo progetto è acquisire una comprensione approfondita, implementare e valutare criticamente Spring Cloud Gateway come soluzione per la gestione dei Gateway API in architetture a microservizi. Per raggiungere questo scopo, sono stati definiti i seguenti sotto-obiettivi specifici:
\begin{itemize}
    \item \textbf{Conduzione di un'indagine (Survey):} Effettuare una ricerca sui concetti fondamentali dei Gateway API e sulle loro funzionalità comuni. Questo aspetto si allinea con la tipologia \textit{Survey} del Project Work, che prevede la ricerca dei principali risultati e caratteristiche di una piattaforma tecnologica.
    \item \textbf{Elaborazione sull'architettura e le funzionalità di Spring Cloud Gateway:} Approfondire l'architettura di SCG, le sue caratteristiche distintive come le route, i predicati e i filtri, e i vantaggi che offre in termini di sicurezza, monitoraggio e resilienza. Questo si inquadra nella tipologia \textit{Elaboration}, concentrandosi su un argomento specifico e il suo stato dell'arte.
    \item \textbf{Sviluppo di un'applicazione pratica (Application):} Realizzare un'implementazione concreta di un Gateway API utilizzando Spring Cloud Gateway. Questo prototipo, disponibile nel repository Git fornito dall'utente, dimostrerà l'applicazione pratica delle funzionalità studiate, come richiesto dalla tipologia \textit{Application} delle linee guida, che prevede l'installazione e l'uso di una tecnologia innovativa attraverso esempi semplici.
    \item \textbf{Analisi critica e identificazione delle limitazioni:} Condurre un'analisi obiettiva di Spring Cloud Gateway, identificando i suoi limiti, le potenziali sfide e gli ostacoli che possono emergere durante la sua adozione. Questa componente è fondamentale per soddisfare il requisito di \textit{Critical Thinking} delle linee guida, che invita a non \textit{innamorarsi} delle nuove tecnologie ma a cogliere anche i loro limiti.
    \item \textbf{Proposta di sviluppi futuri:} Sulla base dell'analisi critica, suggerire possibili evoluzioni e miglioramenti per il progetto implementato o per la tecnologia stessa, in linea con il requisito di indicare \textit{quali saranno gli sviluppi futuri che dovranno essere affrontati per ovviare i limiti evidenziati}.
\end{itemize}
L'esplicita correlazione degli obiettivi del progetto con le tipologie di Project Work definite nelle linee guida (Survey, Elaboration, Application) dimostra una comprensione approfondita dei requisiti dell'incarico. Ciò non si limita alla mera realizzazione tecnica, ma si estende alla capacità di inquadrare il lavoro all'interno di un rigoroso contesto accademico, evidenziando una competenza che va oltre la semplice implementazione.

\section{Struttura del Rapporto}

Il presente rapporto è strutturato per fornire una trattazione completa e progressiva dell'argomento, partendo dai concetti fondamentali fino all'implementazione pratica e all'analisi critica.
\begin{itemize}
    \item \textbf{Capitolo 1: Introduzione} --- Definisce il contesto dei Gateway API nelle architetture a microservizi, le motivazioni alla base della scelta di Spring Cloud Gateway e gli obiettivi specifici del progetto.
    \item \textbf{Capitolo 2: Concetti Fondamentali dei Gateway API} --- Illustra la definizione e il ruolo di un Gateway API, esplorando le sue funzionalità comuni e i vantaggi architetturali che comporta.
    \item \textbf{Capitolo 3: Spring Cloud Gateway: Caratteristiche e Architettura} --- Approfondisce i principi fondamentali di Spring Cloud Gateway, descrivendo i suoi componenti chiave (route, predicati, filtri) e le funzionalità avanzate relative a sicurezza, monitoraggio e resilienza.
    \item \textbf{Capitolo 4: Vantaggi e Casi d'Uso di Spring Cloud Gateway} --- Sintetizza i benefici architetturali e operativi derivanti dall'adozione di SCG e presenta scenari applicativi tipici in cui questa tecnologia eccelle.
    \item \textbf{Capitolo 5: Implementazione Pratica: Il Progetto API Gateway} --- Descrive in dettaglio l'architettura e il design del Gateway implementato, fornendo esempi concreti di configurazione, codice e dimostrazioni pratiche delle funzionalità.
    \item \textbf{Capitolo 6: Analisi Critica e Limitazioni} --- Affronta gli ostacoli e le difficoltà incontrate durante lo studio e l'implementazione, analizzando i limiti di applicabilità di Spring Cloud Gateway e proponendo aree per futuri sviluppi.
    \item \textbf{Capitolo 7: Conclusioni e Sviluppi Futuri} --- Riepiloga i risultati principali del progetto, evidenziando i contributi apportati e delineando le prospettive future per la ricerca e l'applicazione di Spring Cloud Gateway.
    \item \textbf{Biblio/Sitografia} --- Elenca tutte le fonti bibliografiche e sitografiche utilizzate per la redazione del rapporto.
\end{itemize}
Questa struttura è stata concepita per guidare il lettore attraverso un percorso logico, dalla teoria alla pratica, fino a una valutazione ponderata, rispondendo in modo esaustivo a tutti i requisiti delle linee guida del Project Work.