\documentclass[a4paper]{book}
\usepackage[T1]{fontenc}
\usepackage[utf8]{inputenc}
\usepackage[italian]{babel}

\usepackage{geometry}
\geometry{
 a4paper,
 total={170mm,257mm},
 left=20mm,
 top=20mm,
}

\usepackage{fancyhdr}
\usepackage{graphicx}
\usepackage{csquotes}

\usepackage[hidelinks]{hyperref}
\hypersetup{
    pdftitle={Studio e Implementazione di un Gateway API con Spring Cloud in Architetture a Microservizi},
    pdfauthor={Andrea Donadello},
    pdfkeywords={API Gateway, Spring Cloud, Microservizi, Architetture}
    colorlinks=true,
    linkcolor=blue,
    urlcolor=blue,
    citecolor=blue,
}

\usepackage{listings}
\usepackage{xcolor}
\usepackage{pgfplots}
\usepackage{biblatex}
\usepackage{lmodern}
\usepackage{float}
\usepackage{enumitem}
\usepackage{array}
\usepackage{booktabs}
\usepackage{caption}  
\usepackage{xurl}
\usepackage{amsmath} 
\usepackage{tabularx}
\usepackage{ragged2e}  
\usepackage{textcomp} 
\usepackage{microtype}
\usepackage{adjustbox} 

\title{Studio e Implementazione di un Gateway API con Spring Cloud in Architetture a Microservizi}
\author{Andrea Donadello} 

\renewcommand{\lstlistlistingname}{Indice del codice}

\widowpenalty=10000
\clubpenalty=10000

\captionsetup{justification=centering, font=small, labelfont=bf}

\pgfplotsset{compat=1.18}

\setlist{itemsep=0pt, topsep=0.5ex}

\definecolor{codegray}{rgb}{0.5,0.5,0.5}
\definecolor{codegreen}{rgb}{0,0.5,0}
\definecolor{codepurple}{rgb}{0.58,0,0.82}
\definecolor{codeblue}{rgb}{0.0,0.2,0.65}
\definecolor{codeorange}{rgb}{0.9,0.3,0}
\definecolor{codekey}{rgb}{0.1,0.1,0.7} 
\definecolor{codestring}{rgb}{0.8,0.2,0.2} 
\definecolor{codecomment}{rgb}{0.45,0.45,0.45}\itshape
\definecolor{codeannotation}{rgb}{0.4,0.4,0.4}
\definecolor{backcolour}{rgb}{0.97,0.97,0.97}
\definecolor{linecolor}{rgb}{0.8,0.8,0.8} 

\lstset{
  basicstyle=\ttfamily\small, 
  breaklines=true,
  breakatwhitespace=true,  
  postbreak=\mbox{\textcolor{red}{$\hookrightarrow$}\space},
}

\lstdefinestyle{BaseStyle}{
    backgroundcolor=\color{backcolour},
    commentstyle=\color{codecomment},
    stringstyle=\color{codestring},
    numberstyle=\tiny\color{codegray},
    basicstyle=\ttfamily\footnotesize,
    breaklines=true,
    captionpos=b,
    frame=single,
    rulecolor=\color{linecolor},
    keepspaces=true,
    numbers=left,
    numbersep=5pt,
    showspaces=false,
    showstringspaces=false,
    showtabs=false,
    tabsize=2,
    columns=fullflexible,
    title=\lstname,
    postbreak=\mbox{\textcolor{red}{$\hookleftarrow$}\space},
    escapeinside={\%*}{*)}
}

\lstdefinelanguage{Java}{
    keywords=[1]{ 
        abstract, continue, for, new, switch, assert, default, goto, package, synchronized,
        boolean, do, if, private, this, break, double, implements, protected, throw,
        byte, else, import, public, throws, case, enum, instanceof, return, transient,
        catch, extends, int, short, try, char, final, interface, static, void,
        class, finally, long, strictfp, volatile, const, float, native, super, while,
        var, record, yield, sealed, non-sealed, permits
    },
    keywordstyle=[1]\color{codeblue}\bfseries,
    keywords=[2]{ 
        true, false, null,
        String, Integer, Double, Boolean, Character, Byte, Short, Long, Float, Object, Void,
        List, ArrayList, Map, HashMap, Set, HashSet, Exception, System, Math,
        Optional, Stream, Runnable, Thread, Class, StringBuilder, StringBuffer
    },
    keywordstyle=[2]\color{codepurple},
    keywords=[3]{ 
        @Override, @Deprecated, @SuppressWarnings, @FunctionalInterface, @SafeVarargs,
        @NonNull, @Nullable, @PostConstruct, @PreDestroy, @Entity, @Table, @Id, @Column,
        @Autowired, @Component, @Service, @Repository, @Controller, @RestController,
        @GetMapping, @PostMapping, @PutMapping, @DeleteMapping, @RequestMapping,
        @Data
    },
    keywordstyle=[3]\color{codeannotation},
    sensitive=true,
    comment=[l]{//},
    morecomment=[s]{/*}{*/},
    morecomment=[s][\color{codegreen!80!black}]{/**}{*/}, % Javadoc
    string=[b]{"},
    morestring=[b]{'}
}

\lstdefinestyle{JavaStyle}{
    style=BaseStyle,
    language=Java,
    literate={->}{{$\rightarrow$}}2 {<-}{{$\leftarrow$}}2
}

\lstdefinelanguage{YAML}{
  keywords=[1]{true, false, null, on, off, yes, no},
  keywordstyle=[1]\color{codeblue},
  identifierstyle=\color{codekey},
  sensitive=false,
  comment=[l]{\#},
  morestring=[b]',
  morestring=[b]",
  literate=
    {>}{{\textcolor{codepurple}>}}1
    {|}{{\textcolor{codepurple}|}}1
    {-}{{{\color{codepurple}-}}}1
    {?}{{{\color{codepurple}?}}}1
    {:}{{{:}}}0,
}

\lstdefinestyle{YAMLStyle}{
    style=BaseStyle,
    language=YAML
}

\lstdefinelanguage{bash}{
    keywords={
        if, then, else, fi, for, in, do, done, while, case, esac, function,
        select, until, echo, exit, return, test, read, unset, export
    },
    keywordstyle=\color{codeblue}\bfseries,
    keywords=[2]{
        ls, cd, pwd, cp, mv, rm, mkdir, rmdir, touch, cat, grep, find,
        sed, awk, tar, gzip, gunzip, chmod, chown, sudo, su, man,
        git, docker, kubectl, java, mvn, gradle, npm, node, python
    },
    keywordstyle=[2]\color{codepurple},
    sensitive=false,
    comment=[l]{\#},
    morestring=[b]",
    morestring=[b]',
    morestring=[s][\color{codestring}]{`}{`} 
}

\lstdefinestyle{BashStyle}{
    style=BaseStyle,
    language=bash,
    identifierstyle=\color{black},
    alsoletter={-,$},
    literate=
      {\\$}{{\textcolor{codepurple}{\$}}}1
      {--}{{\textcolor{codeorange}{--}}}2
      {-}{{\textcolor{codeorange}{-}}}1
      {~}{{\raise.5ex\hbox{\texttildelow}}}1
}

\newenvironment{abstract}%
  {\clearpage\null\vfill\begin{center}%
    \bfseries\abstractname\end{center}}%
  {\vfill\null}
  
\addbibresource{references.bib}

\begin{document}

    \pagestyle{plain}
    \maketitle

    \begin{abstract}
    Il presente lavoro di progetto universitario si concentra sull'analisi approfondita e sull'implementazione pratica di un Gateway API utilizzando Spring Cloud Gateway. \\
    In un'era di trasformazione digitale rapida, la gestione strategica delle API è emersa come un fattore critico per le aziende che adottano architetture a microservizi, dove la complessità della comunicazione inter-servizio, della sicurezza e del routing può diventare proibitiva senza un punto di ingresso centralizzato. \\
    Questo studio si propone di esplorare Spring Cloud Gateway come soluzione robusta e reattiva per affrontare tali sfide. \\

    La metodologia adottata per questo progetto combina elementi di indagine (\textit{Survey}), elaborazione (\textit{Elaboration}) e applicazione pratica (\textit{Application}). \\
    È stata condotta un'indagine sui concetti fondamentali dei Gateway API e sulle loro funzionalità comuni, seguita da un'elaborazione dettagliata dell'architettura e delle caratteristiche specifiche di Spring Cloud Gateway. \\
    Il cuore del progetto consiste nell'implementazione di un prototipo di Gateway API, disponibile nel repository Git fornito (\url{https://github.com/ddonazz/api-gateway}), che dimostra funzionalità chiave quali il routing dinamico, la gestione centralizzata della sicurezza (autenticazione e autorizzazione), la limitazione del tasso di richieste (rate limiting) e la gestione delle eccezioni. \\ 

    I risultati ottenuti evidenziano i significativi vantaggi di Spring Cloud Gateway, tra cui una maggiore sicurezza, un'accelerazione dell'innovazione e tempi di commercializzazione ridotti, una migliore esperienza utente e una conformità normativa più efficiente. \\
    Tuttavia, l'analisi critica ha anche rivelato sfide intrinseche all'adozione di un Gateway API, come l'aumento della complessità architetturale e il potenziale di introduzione di latenza aggiuntiva. \\
    Il lavoro si conclude con una discussione sui contributi del progetto e sulle prospettive future per superare le limitazioni identificate, proponendo sviluppi che potrebbero ulteriormente migliorare la robustezza e l'applicabilità di tali soluzioni.
\end{abstract}

    \tableofcontents

    \listoftables 

    \lstlistoflistings

    \chapter{Introduzione}
    \section{Contesto e Motivazioni}

Il panorama dello sviluppo software contemporaneo è profondamente influenzato dall'adozione sempre più diffusa delle architetture a microservizi. Questo modello, che scompone le applicazioni monolitiche in un insieme di servizi piccoli, autonomi e accoppiati in modo lasco, offre numerosi vantaggi in termini di scalabilità, resilienza e agilità nello sviluppo. Tuttavia, l'adozione dei microservizi introduce anche nuove sfide significative, in particolare per quanto riguarda la gestione della comunicazione tra i servizi, la sicurezza, il monitoraggio e il routing delle richieste. Man mano che il numero di microservizi aumenta, la complessità di gestire le interazioni dirette tra client e servizi backend può diventare insostenibile, portando a problemi come la logica client frammentata e la difficoltà nella gestione delle preoccupazioni trasversali.

In questo contesto, il Gateway API emerge come un componente architetturale cruciale, fungendo da \enquote{edge service} e da \enquote{reverse proxy} che centralizza l'esposizione delle API e gestisce le preoccupazioni trasversali. Questo approccio risolve il problema \enquote{N+1}, dove i client dovrebbero altrimenti effettuare chiamate separate a N servizi backend, aggregando le richieste e semplificando l'interazione. La gestione strategica delle API è, infatti, un fattore critico per le aziende che perseguono la trasformazione digitale, distinguendo i leader di mercato dai ritardatari.

La scelta di Spring Cloud Gateway (SCG) come tecnologia centrale per questo studio è giustificata dalla sua rilevanza all'interno dell'ecosistema Spring, ampiamente adottato nello sviluppo di applicazioni aziendali. SCG è una soluzione leggera e reattiva, costruita su Spring WebFlux e Project Reactor, che la rende particolarmente adatta per carichi di lavoro ad alta produttività e bassa latenza. Il presente Project Work si focalizza sullo studio di questa \enquote{nuova tecnologia}, come richiesto dalle linee guida, fornendo dettagli implementativi e un'analisi critica della sua applicazione pratica. La necessità di un punto di ingresso centralizzato per le API è una conseguenza diretta dell'adozione dei microservizi; senza un Gateway API, la gestione delle interazioni tra client e servizi backend, la sicurezza e il monitoraggio diventerebbero estremamente complessi e frammentati, ostacolando l'efficienza e la scalabilità del sistema distribuito.

\section{Obiettivi del Progetto}

L'obiettivo primario di questo progetto è acquisire una comprensione approfondita, implementare e valutare criticamente Spring Cloud Gateway come soluzione per la gestione dei Gateway API in architetture a microservizi. Per raggiungere questo scopo, sono stati definiti i seguenti sotto-obiettivi specifici:
\begin{itemize}
    \item \textbf{Conduzione di un'indagine (Survey):} Effettuare una ricerca sui concetti fondamentali dei Gateway API e sulle loro funzionalità comuni. Questo aspetto si allinea con la tipologia \enquote{Survey} del Project Work, che prevede la ricerca dei principali risultati e caratteristiche di una piattaforma tecnologica.
    \item \textbf{Elaborazione sull'architettura e le funzionalità di Spring Cloud Gateway:} Approfondire l'architettura di SCG, le sue caratteristiche distintive come le route, i predicati e i filtri, e i vantaggi che offre in termini di sicurezza, monitoraggio e resilienza. Questo si inquadra nella tipologia \enquote{Elaboration}, concentrandosi su un argomento specifico e il suo stato dell'arte.
    \item \textbf{Sviluppo di un'applicazione pratica (Application):} Realizzare un'implementazione concreta di un Gateway API utilizzando Spring Cloud Gateway. Questo prototipo, disponibile nel repository Git fornito dall'utente, dimostrerà l'applicazione pratica delle funzionalità studiate, come richiesto dalla tipologia \enquote{Application} delle linee guida, che prevede l'installazione e l'uso di una tecnologia innovativa attraverso esempi semplici.
    \item \textbf{Analisi critica e identificazione delle limitazioni:} Condurre un'analisi obiettiva di Spring Cloud Gateway, identificando i suoi limiti, le potenziali sfide e gli ostacoli che possono emergere durante la sua adozione. Questa componente è fondamentale per soddisfare il requisito di \enquote{Critical Thinking} delle linee guida, che invita a non \enquote{innamorarsi} delle nuove tecnologie ma a cogliere anche i loro limiti.
    \item \textbf{Proposta di sviluppi futuri:} Sulla base dell'analisi critica, suggerire possibili evoluzioni e miglioramenti per il progetto implementato o per la tecnologia stessa, in linea con il requisito di indicare \enquote{quali saranno gli sviluppi futuri che dovranno essere affrontati per ovviare i limiti evidenziati}.
\end{itemize}
L'esplicita correlazione degli obiettivi del progetto con le tipologie di Project Work definite nelle linee guida (Survey, Elaboration, Application) dimostra una comprensione approfondita dei requisiti dell'incarico. Ciò non si limita alla mera realizzazione tecnica, ma si estende alla capacità di inquadrare il lavoro all'interno di un rigoroso contesto accademico, evidenziando una competenza che va oltre la semplice implementazione.

\section{Struttura del Rapporto}

Il presente rapporto è strutturato per fornire una trattazione completa e progressiva dell'argomento, partendo dai concetti fondamentali fino all'implementazione pratica e all'analisi critica.
\begin{itemize}
    \item \textbf{Capitolo 1: Introduzione} --- Definisce il contesto dei Gateway API nelle architetture a microservizi, le motivazioni alla base della scelta di Spring Cloud Gateway e gli obiettivi specifici del progetto.
    \item \textbf{Capitolo 2: Concetti Fondamentali dei Gateway API} --- Illustra la definizione e il ruolo di un Gateway API, esplorando le sue funzionalità comuni e i vantaggi architetturali che comporta.
    \item \textbf{Capitolo 3: Spring Cloud Gateway: Caratteristiche e Architettura} --- Approfondisce i principi fondamentali di Spring Cloud Gateway, descrivendo i suoi componenti chiave (route, predicati, filtri) e le funzionalità avanzate relative a sicurezza, monitoraggio e resilienza.
    \item \textbf{Capitolo 4: Vantaggi e Casi d'Uso di Spring Cloud Gateway} --- Sintetizza i benefici architetturali e operativi derivanti dall'adozione di SCG e presenta scenari applicativi tipici in cui questa tecnologia eccelle.
    \item \textbf{Capitolo 5: Implementazione Pratica: Il Progetto API Gateway} --- Descrive in dettaglio l'architettura e il design del Gateway implementato, fornendo esempi concreti di configurazione, codice e dimostrazioni pratiche delle funzionalità.
    \item \textbf{Capitolo 6: Analisi Critica e Limitazioni} --- Affronta gli ostacoli e le difficoltà incontrate durante lo studio e l'implementazione, analizzando i limiti di applicabilità di Spring Cloud Gateway e proponendo aree per futuri sviluppi.
    \item \textbf{Capitolo 7: Conclusioni e Sviluppi Futuri} --- Riepiloga i risultati principali del progetto, evidenziando i contributi apportati e delineando le prospettive future per la ricerca e l'applicazione di Spring Cloud Gateway.
    \item \textbf{Biblio/Sitografia} --- Elenca tutte le fonti bibliografiche e sitografiche utilizzate per la redazione del rapporto.
\end{itemize}
Questa struttura è stata concepita per guidare il lettore attraverso un percorso logico, dalla teoria alla pratica, fino a una valutazione ponderata, rispondendo in modo esaustivo a tutti i requisiti delle linee guida del Project Work.

    \chapter[Concetti Fondamentali]{Concetti Fondamentali dei Gateway API}
    \section{Cos'è un API Gateway e il suo Ruolo nelle Architetture a Microservizi}

Un API Gateway è un componente architetturale che funge da singolo punto di ingresso per tutte le richieste dei client verso un sistema di microservizi backend. In essenza, agisce come un \enquote{reverse proxy} o un \enquote{edge service} che si interpone tra i client e i servizi backend. Questo modello risolve una problematica comune nelle architetture a microservizi, nota come \enquote{N+1 problem}, dove un client dovrebbe altrimenti conoscere e interagire direttamente con N servizi backend distinti per soddisfare una singola richiesta complessa. L'API Gateway aggrega queste interazioni, presentando un'unica interfaccia semplificata ai client.

Il ruolo di un API Gateway è multifunzionale e critico per la gestione efficace di sistemi distribuiti. Esso non si limita al semplice routing delle richieste al servizio appropriato, ma centralizza anche una serie di preoccupazioni trasversali che altrimenti dovrebbero essere implementate in ogni singolo microservizio. Questa centralizzazione è fondamentale per mantenere la coerenza, ridurre la ridondanza del codice e accelerare lo sviluppo.

La funzione di un API Gateway come punto di ingresso unificato semplifica intrinsecamente l'integrazione dei client e disaccoppia i client dall'evoluzione dei servizi di backend. Se i client fossero costretti a interagire direttamente con ogni microservizio, dovrebbero essere a conoscenza degli indirizzi specifici, dei protocolli e delle modifiche di ciascuno. Un Gateway API astrae questa complessità, offrendo un'interfaccia coerente. Questa separazione significa che le modifiche ai servizi di backend (ad esempio, la fusione o la divisione di servizi, o la migrazione di tecnologie) possono essere implementate senza richiedere aggiornamenti sul lato client. Ciò riduce significativamente il sovraccarico di manutenzione e accelera i cicli di sviluppo, poiché i team possono concentrarsi sulla logica di business senza preoccuparsi delle implicazioni per i client.

\section{Funzionalità Comuni dei Gateway API}

Un Gateway API robusto e completo offre una vasta gamma di funzionalità essenziali per la gestione efficiente e sicura delle API in un ambiente a microservizi. Queste funzionalità sono progettate per affrontare le sfide intrinseche dei sistemi distribuiti e per centralizzare le preoccupazioni trasversali.

Le funzionalità comuni includono:
\begin{itemize}
    \item \textbf{Routing:} La capacità fondamentale di dirigere le richieste in ingresso al servizio backend corretto in base a regole predefinite, come il percorso dell'URL, gli header o i parametri della richiesta.
    \item \textbf{Autenticazione e Autorizzazione:} Applicazione centralizzata delle politiche di sicurezza. Questo include la gestione dell'autenticazione (ad esempio, Basic Auth, OAuth2, OpenID Connect) e dell'autorizzazione (controlli di accesso basati sui ruoli). La funzionalità di Single Sign-On (SSO) è spesso integrata per semplificare l'accesso tra diverse applicazioni.
    \item \textbf{Limitazione del Tasso (Rate Limiting):} Controllo del numero di richieste che un client può effettuare in un determinato periodo di tempo per prevenire abusi, attacchi Denial of Service (DoS) e garantire un uso equo delle risorse.
    \item \textbf{Bilanciamento del Carico (Load Balancing) e Tolleranza ai Guasti (Fault Tolerance):} Distribuzione uniforme del traffico in ingresso tra più istanze di un servizio backend per migliorare le prestazioni e la disponibilità, garantendo che il sistema rimanga reattivo anche in caso di guasti o sovraccarichi di singoli servizi.
    \item \textbf{Monitoraggio e Logging:} Raccolta centralizzata di metriche e log relativi al traffico API, alle prestazioni e agli errori, fornendo visibilità sull'operatività del sistema e facilitando il debugging.
    \item \textbf{Trasformazione di Protocollo e Formato Dati:} Capacità di convertire richieste e risposte tra diversi protocolli (ad esempio, HTTP a gRPC) o formati di dati (ad esempio, JSON a XML), facilitando l'integrazione tra sistemi eterogenei.
    \item \textbf{Versionamento delle API:} Gestione di più versioni di un'API, consentendo agli sviluppatori di introdurre nuove funzionalità o modifiche senza interrompere i client esistenti.
    \item \textbf{Caching:} Memorizzazione di risposte a richieste frequenti per ridurre il carico sui servizi backend e migliorare i tempi di risposta.
    \item \textbf{Circuit Breaker (Interruttore di Circuito):} Un pattern di resilienza che previene i guasti a cascata in un sistema distribuito, isolando i servizi che non rispondono e fornendo risposte di fallback.
    \item \textbf{Header di Sicurezza:} Applicazione automatica di header di sicurezza (ad esempio, Cache-Control, X-Content-Type-Options, Strict-Transport-Security) per rafforzare la postura di sicurezza complessiva.
\end{itemize}

L'aggregazione di queste preoccupazioni trasversali all'interno del Gateway API le trasforma da responsabilità individuali di ciascun servizio in politiche centralizzate e gestibili. Questo approccio riduce il carico di sviluppo sui team, che possono concentrarsi sulla creazione di valore di business piuttosto che sulla ricostruzione di componenti infrastrutturali. Il risultato è un'accelerazione dell'innovazione e del tempo di commercializzazione, oltre a una conformità più efficiente e un'esperienza cliente superiore, derivanti dall'applicazione coerente di best practice di sicurezza e gestione del traffico.

\begin{table}[htbp]
\centering
\caption{Funzionalità Comuni di un API Gateway (con librerie standard)}
\renewcommand{\arraystretch}{1.5}
\label{tab:funzionalita_api_gateway_standard}
\begin{tabularx}{\linewidth}{
    >{\raggedright\arraybackslash}p{0.22\linewidth} 
    >{\raggedright\arraybackslash}X                 
    >{\raggedright\arraybackslash}X                 
}
\toprule
\textbf{Funzionalità} & \textbf{Descrizione} & \textbf{Beneficio Principale} \\
\midrule
Routing & Inoltra le richieste in ingresso al servizio backend appropriato. & Semplifica l'integrazione client e la gestione del traffico. \\
Autenticazione \& Autorizzazione & Applica politiche di sicurezza centralizzate per l'accesso alle API. & Migliora la sicurezza e la conformità, riduce il rischio. \\
Limitazione del Tasso (Rate Limiting) & Controlla il numero di richieste per prevenire abusi e sovraccarichi. & Garantisce stabilità del sistema e equità d'uso. \\
Bilanciamento del Carico & Distribuisce il traffico tra le istanze del servizio per ottimizzare le prestazioni. & Aumenta la disponibilità e la scalabilità. \\
Tolleranza ai Guasti & Gestisce i fallimenti dei servizi backend per mantenere la disponibilità. & Aumenta la resilienza del sistema. \\
Monitoraggio \& Logging & Raccoglie dati sul traffico API, le prestazioni e gli errori. & Fornisce visibilità operativa e facilita il debugging. \\
Trasformazione Protocollo/Dati & Converte richieste/risposte tra diversi formati o protocolli. & Facilita l'integrazione con sistemi eterogenei. \\
Versionamento API & Gestisce più versioni di un'API contemporaneamente. & Consente evoluzione delle API senza interrompere i client esistenti. \\
Caching & Memorizza le risposte per richieste frequenti. & Riduce la latenza e il carico sui servizi backend. \\
Circuit Breaker (Interruttore di Circuito) & Isola i servizi problematici per prevenire guasti a cascata. & Migliora la resilienza del sistema in ambienti distribuiti. \\
Header di Sicurezza & Applica automaticamente header HTTP per rafforzare la sicurezza. & Migliora la postura di sicurezza complessiva. \\
\bottomrule
\end{tabularx}
\end{table}

    \chapter[Spring Cloud Gateway]{Spring Cloud Gateway: Caratteristiche e Architettura}
    \section{Panoramica e Principi Fondamentali}

Spring Cloud Gateway (SCG) si presenta come un Gateway API leggero e reattivo, costruito sopra il framework Spring. \\
È stato progettato per fornire un metodo semplice ma efficace per instradare le richieste alle API e per gestire le preoccupazioni trasversali quali sicurezza, monitoraggio/metriche e resilienza. \\
La sua architettura è saldamente radicata su tecnologie moderne dell'ecosistema Spring: Spring Framework 5, Project Reactor e Spring Boot 2.0. 
Questa fondazione è cruciale, poiché conferisce a SCG la sua natura reattiva e non bloccante, un aspetto distintivo che lo rende particolarmente adatto per gestire un elevato numero di richieste concorrenti con bassa latenza. \\

La scelta di basare SCG su Project Reactor e Spring WebFlux non è un mero dettaglio tecnico, ma una decisione architetturale fondamentale che lo distingue dai gateway tradizionali basati su modelli di programmazione bloccanti. \\
I modelli di programmazione reattiva sono intrinsecamente progettati per gestire un gran numero di connessioni concorrenti con un numero ridotto di thread, ottimizzando l'utilizzo delle risorse e migliorando le prestazioni sotto carichi pesanti. 
Questo è particolarmente vantaggioso in ambienti a microservizi, dove la capacità di gestire efficacemente un flusso elevato di traffico è essenziale per garantire \textit{carichi di lavoro a bassa latenza e alta produttività}.  \\

Il ciclo di vita di una richiesta all'interno di Spring Cloud Gateway segue un flusso ben definito:
\begin{enumerate}[label=\arabic*.]
    \item \textbf{Client Request:} Una richiesta viene inviata dal client al Gateway API.
    \item \textbf{Gateway Handler Mapping:} Il Gateway determina se la richiesta corrisponde a una route definita.
    \item \textbf{Gateway Web Handler:} Se una corrispondenza viene trovata, la richiesta viene inoltrata a questo handler.
    \item \textbf{Filter Chain:} L'handler esegue la richiesta attraverso una catena di filtri specifici per quella richiesta. Questi filtri possono modificare la richiesta prima che venga inviata al servizio di destinazione (pre-filtri) o la risposta che ritorna al client (post-filtri).
    \item \textbf{Downstream Service:} La richiesta modificata viene inviata al servizio backend appropriato.
    \item \textbf{Response:} La risposta dal servizio backend ritorna attraverso la catena di filtri (post-filtri) prima di essere inviata al client.
\end{enumerate}
Questo modello di elaborazione basato su filtri e route consente una grande flessibilità e modularità nella gestione del traffico API.

\section{Componenti Chiave: Route, Predicati e Filtri}

Spring Cloud Gateway si basa su tre componenti fondamentali per la sua operatività: Route, Predicati e Filtri. La modularità di questi elementi consente una configurazione dichiarativa e un'estensibilità notevole, permettendo di gestire logiche di routing complesse e preoccupazioni trasversali con grande flessibilità.
\begin{description}
    \item[Route:] Una Route è il \textit{blocco di costruzione fondamentale} del gateway. È definita da un ID univoco, un URI di destinazione (il servizio backend a cui la richiesta deve essere inoltrata), una collezione di predicati e una collezione di filtri. Una route viene considerata \textit{matched} (corrispondente) se l'operazione logica AND su tutti i suoi predicati restituisce true.
    \lstinputlisting[style=YAMLStyle, caption=Esempio di configurazione di una route in YAML]{code/route-configuration.yml}
    \item[Predicate:] I Predicati sono funzioni booleane (Java 8 Function Predicate) che permettono di far corrispondere le route a qualsiasi attributo della richiesta HTTP. Questo include header, parametri, metodi HTTP, host, e persino il tempo. È possibile combinare più predicati con operatori logici AND per creare regole di routing altamente specifiche.
    Esempi comuni di predicati includono: Path, Host, Method, Query, Header, Cookie, RemoteAddr, After, Before, Between. I predicati After, Before e Between sono particolarmente utili per scenari come le finestre di manutenzione, permettendo di instradare il traffico in base a intervalli temporali specifici.
    \item[Filter:] I Filtri forniscono un meccanismo per modificare le richieste HTTP in ingresso e le risposte HTTP in uscita. Esistono due tipi principali di filtri:
    \begin{itemize}
        \item \textbf{GatewayFilter:} Applicati a una specifica route.
        \item \textbf{GlobalFilter:} Applicati a tutte le route. Spring Cloud Gateway include molti filtri predefiniti (ad esempio, AddRequestHeader, AddResponseHeader, RateLimiter, RewritePath). È anche possibile implementare filtri personalizzati per esigenze di business uniche. Ad esempio, un filtro personalizzato potrebbe essere utilizzato per aggiungere un timestamp alla richiesta in ingresso prima che venga inoltrata (AddRequestTimeHeaderPreFilter) o per simulare una validazione e bloccare la richiesta se un header di autorizzazione non è presente.
    \end{itemize}
\end{description}

L'enfasi ricorrente su Route, Predicati e Filtri sottolinea la loro centralità nel potere di Spring Cloud Gateway. 
La capacità di definire le route in modo dichiarativo, di combinare i predicati per una corrispondenza granulare e di applicare i filtri per la modifica delle richieste/risposte significa che le politiche di gestione delle API, anche le più complesse, possono essere costruite a partire da blocchi più piccoli e riutilizzabili. 
Questa modularità rende il gateway altamente configurabile ed estensibile, consentendo agli sviluppatori di aggiungere facilmente nuove funzionalità o di implementare filtri personalizzati per requisiti specifici. 
Questo principio di design è fondamentale per la sua adattabilità in diversi ambienti a microservizi.

\begin{table}[ht!]
\small
\centering
\caption{Esempi di Predicati di Route in Spring Cloud Gateway}
\renewcommand{\arraystretch}{1.5}
\label{tab:predicati_scg}
\begin{tabularx}{\linewidth}{%
    >{\RaggedRight\arraybackslash}p{0.15\linewidth}
    >{\RaggedRight\arraybackslash}X
    >{\RaggedRight\arraybackslash}p{0.20\linewidth}
    >{\RaggedRight\arraybackslash}X
}
\toprule
\textbf{Predicato} & \textbf{Descrizione} & \textbf{Parametri Esempio} & \textbf{Esempio di Configurazione} \\
\midrule
Path & Corrisponde al percorso dell'URL della richiesta. & \lstinline{/api/users/**} & \lstinline{predicates: - Path=/api/users/**} \\
Host & Corrisponde all'header Host della richiesta. & \lstinline{*.example.com} & \lstinline{predicates: - Host=*.example.com} \\
Method & Corrisponde al metodo HTTP della richiesta. & GET, POST & \lstinline{predicates: - Method=GET,POST} \\
Query & Corrisponde a un parametro di query specifico. & param=value & \lstinline{predicates: - Query=param,value} \\
Header & Corrisponde a un header HTTP specifico e al suo valore. & \lstinline|X-Request-Id, \d+| & \lstinline|predicates: - Header=X-Request-Id, \d+| \\ 
After & Corrisponde alle richieste che avvengono dopo una data e ora specificate. & 2017-01-20T... & \lstinline|predicates: - After=2017-01-20T17:42:47.789Z[UTC]| \\ 
Before & Corrisponde alle richieste che avvengono prima di una data e ora specificate. & 2017-01-21T... & \lstinline|predicates: - Before=2017-01-21T17:42:47.789Z[UTC]| \\
Between & Corrisponde alle richieste che avvengono tra due date e ore specificate. & datetime1, datetime2 & \lstinline|predicates: - Between=..., ...| \\
RemoteAddr & Corrisponde all'indirizzo IP remoto della richiesta. & 192.168.1.1/24 & \lstinline{predicates: - RemoteAddr=192.168.1.1/24} \\
\bottomrule
\end{tabularx}
\end{table}

\begin{table}[ht!]
\small
\centering
\caption{Esempi di Filtri in Spring Cloud Gateway}
\renewcommand{\arraystretch}{1.5}
\label{tab:filtri_scg_}
\begin{tabularx}{\linewidth}{%
    >{\RaggedRight\arraybackslash}p{0.2\linewidth} 
    >{\RaggedRight\arraybackslash}p{0.12\linewidth} 
    >{\RaggedRight\arraybackslash}X                 
    >{\RaggedRight\arraybackslash}X                 
}
\toprule
\textbf{Filtro} & \textbf{Tipo} & \textbf{Descrizione} & \textbf{Caso d'Uso Esempio} \\
\midrule
AddRequestHeader & GatewayFilter & Aggiunge un header alla richiesta in uscita. & \lstinline{filters: - AddRequestHeader=X-Request-Foo, Bar} \\
AddResponseHeader & GatewayFilter & Aggiunge un header alla risposta in uscita. & \lstinline{filters: - AddResponseHeader=X-Response-Bye, Bye} \\
StripPrefix & GatewayFilter & Rimuove un prefisso dal percorso della richiesta. & \lstinline{filters: - StripPrefix=1} per /api/v1/users $\rightarrow$ /users \\
RewritePath & GatewayFilter & Riscrive il percorso della richiesta usando un'espressione regolare. & \lstinline{filters: - RewritePath=/foo/(?<segment>.*), /\$\{segment\}} \\
RequestRateLimiter & GatewayFilter & Limita il tasso di richieste per utente/IP. & \lstinline{filters: - RequestRateLimiter=} (con config.) \\
CircuitBreaker & GatewayFilter & Implementa un pattern Circuit Breaker per la resilienza. & \lstinline{filters: - CircuitBreaker=myServiceCircuit} \\
SecureHeaders & GlobalFilter & Applica globalmente header di sicurezza HTTP. & \lstinline{tanzu: api-gateway: secure-headers: deactivated: false} \\
CustomGlobal\-ExceptionHandler & GlobalFilter & Gestisce eccezioni a livello globale per risposte uniformi. & Intercetta HttpClientErrorException per risposte di errore consistenti \\
Authorization Filter & GlobalFilter & Filtro personalizzato per autenticazione/autorizzazione. & Verifica token OAuth2 o credenziali Basic Auth \\
AddRequestTime\-Header\-PreFilter & GlobalFilter & Filtro personalizzato che aggiunge un timestamp alla richiesta. & \lstinline{filters: - AddRequestTimeHeaderPreFilter} \\
\bottomrule
\end{tabularx}
\end{table}

\section{Funzionalità Avanzate: Sicurezza, Monitoraggio, Resilienza}

Oltre alle capacità fondamentali di routing e trasformazione, Spring Cloud Gateway eccelle nella gestione di funzionalità avanzate che sono cruciali per la robustezza e la sicurezza delle architetture a microservizi. Queste includono la sicurezza centralizzata, il monitoraggio e le metriche, e la resilienza del sistema.
\begin{description}
    \item[Sicurezza:] SCG centralizza le preoccupazioni di sicurezza, riducendo la necessità di implementare logiche di sicurezza in ogni singolo microservizio. Questo include:
    \begin{itemize}
        \item \textit{Autenticazione e Autorizzazione:} Supporta vari meccanismi di autenticazione come Basic Auth e OAuth2/OpenID Connect. Le funzionalità di Single Sign-On (SSO) sono particolarmente rilevanti, in quanto possono essere configurate una sola volta a livello di gateway, eliminando la necessità di implementazioni diverse per ogni sistema backend. Ciò consente un controllo degli accessi basato sui ruoli e semplifica notevolmente la conformità normativa, riducendo i costi di conformità fino al 40\% per le organizzazioni con controlli di sicurezza centralizzati.
        \item \textit{Header di Sicurezza:} L'applicazione automatica di header di sicurezza (es. Cache-Control, X-Content-Type-Options, Strict-Transport-Security) a livello globale per tutte le route è una best practice che rafforza la postura di sicurezza complessiva.
        \item \textit{Gestione Globale delle Eccezioni:} Implementare una gestione globale delle eccezioni in SCG è essenziale per catturare e gestire in modo uniforme gli errori che si verificano all'interno del gateway. Questo previene risposte di errore frammentate e incoerenti, migliorando l'esperienza utente e facilitando il debugging.
    \end{itemize}
    \item[Monitoraggio/Metriche:] SCG può essere integrato con strumenti di monitoraggio e logging esterni come Splunk o ELK Stack. Questa integrazione fornisce una visibilità centralizzata sul traffico API, sulle prestazioni e sugli errori. La capacità di registrare ogni eccezione per analisi future è una best practice fondamentale.
    \item[Resilienza:] La resilienza è vitale in ambienti distribuiti per prevenire i guasti a cascata. SCG offre funzionalità come:
    \begin{itemize}
        \item \textit{Integrazione con Circuit Breaker:} L'integrazione con circuit breaker come Hystrix (o Resilience4j in versioni più recenti) consente di isolare i servizi che non rispondono, evitando che un singolo guasto si propaghi attraverso l'intero sistema.
        \item \textit{Gestione Intelligente del Traffico e Tolleranza ai Guasti:} SCG offre funzionalità avanzate di gestione del traffico e alta disponibilità che possono mantenere prestazioni consistenti anche durante i picchi di traffico. Questo include il bilanciamento del carico e la gestione dei fallimenti.
    \end{itemize}
\end{description}
La centralizzazione di funzionalità avanzate come la sicurezza e la resilienza a livello di gateway le trasforma da correzioni reattive in salvaguardie architetturali proattive. Questo approccio migliora significativamente la robustezza complessiva del sistema e la sua postura di conformità. Gestendo questi aspetti in modo centralizzato, il sistema diventa più resistente ai guasti e agli attacchi, e la conformità normativa è più facile da gestire, riducendo potenziali sanzioni. Questo non si limita alla mera funzionalità, ma si estende all'impatto strategico di tali funzionalità sull'integrità operativa del sistema e sui risultati di business.

    \chapter[Vantaggi e Casi d'Uso]{Vantaggi e Casi d'Uso di Spring Cloud Gateway}
    \section{Benefici Architetturali e Operativi}

L'adozione di Spring Cloud Gateway in un'architettura a microservizi porta a una serie di benefici significativi, sia a livello architetturale che operativo. Questi vantaggi non sono solo tecnici, ma si traducono direttamente in risultati di business misurabili.

\begin{itemize}
    \item \textbf{Integrazione Client Semplificata:} I client interagiscono con un unico endpoint esposto dal Gateway API, semplificando la loro logica applicativa e riducendo la complessità di dover conoscere e gestire più servizi backend.
    \item \textbf{Sicurezza Migliorata:} La centralizzazione delle politiche di sicurezza, inclusa l'autenticazione, l'autorizzazione e l'applicazione di header di sicurezza, riduce il rischio di incidenti di sicurezza e i costi di conformità. Le organizzazioni con pratiche mature di sicurezza API registrano il 60\% in meno di incidenti di sicurezza.
    \item \textbf{Innovazione Accelerata e Tempo di Commercializzazione Ridotto:} I team di sviluppo possono concentrarsi sulla logica di business principale piuttosto che sulla ricostruzione di componenti infrastrutturali come la sicurezza o la gestione del traffico. La ricerca McKinsey indica che le organizzazioni con una gestione API semplificata rilasciano nuove funzionalità 2-3 volte più velocemente della concorrenza.
    \item \textbf{Prestazioni e Scalabilità Migliorate:} L'architettura reattiva e non bloccante di SCG, basata su Spring WebFlux e Project Reactor, consente di gestire un gran numero di richieste concorrenti con un utilizzo efficiente delle risorse. Funzionalità come la gestione intelligente del traffico, il bilanciamento del carico e la tolleranza ai guasti contribuiscono a mantenere prestazioni elevate. SCG ha dimostrato risultati impressionanti, con tempi di risposta API fino a 6 volte più veloci, latenza ridotta fino al 50\% e oltre l'80\% di riduzione dei tassi di errore rispetto a opzioni concorrenti.
    \item \textbf{Migliore Monitoraggio e Visibilità:} La centralizzazione dei log e delle metriche per le preoccupazioni trasversali fornisce una visibilità completa sul comportamento delle API e facilita l'identificazione e la risoluzione dei problemi.
    \item \textbf{Conformità Cost-Effective:} I controlli di sicurezza centralizzati consentono di ridurre i costi di conformità fino al 40\%. SCG implementa questi controlli in modo coerente, creando audit trail e applicando politiche di sicurezza con un overhead minimo.
    \item \textbf{Versionamento API e Compatibilità Retroattiva:} SCG può gestire più versioni di un'API, consentendo agli sviluppatori di introdurre nuove funzionalità o apportare modifiche senza interrompere i client esistenti, garantendo una transizione più fluida e riducendo il rischio di interruzioni del servizio.
\end{itemize}

I vantaggi tecnici di Spring Cloud Gateway si traducono direttamente in risultati di business tangibili, come la riduzione dei costi operativi, una risposta più rapida al mercato e un'esperienza cliente superiore. Questo collegamento tra capacità tecniche e obiettivi strategici di business è fondamentale per comprendere il valore di SCG, in linea con il requisito di spiegare \enquote{quali vantaggi produce questa tecnologia in quali settori applicativi}.

\section{Scenari Applicativi Tipici}

La versatilità di Spring Cloud Gateway lo rende idoneo per un'ampia gamma di scenari applicativi, affrontando diverse sfide architetturali in contesti moderni e legacy.

\begin{itemize}
    \item \textbf{Orchestrazione di Microservizi:} Il caso d'uso più comune è l'orchestrazione del traffico in ingresso verso un'architettura a microservizi. SCG agisce come il punto di ingresso che instrada le richieste ai vari servizi backend, gestendo le preoccupazioni trasversali in modo centralizzato.
    \item \textbf{Integrazione di Sistemi Legacy:} SCG può fungere da ponte tra sistemi legacy e applicazioni moderne. Grazie alle sue capacità di trasformazione di protocollo e formato dati (ad esempio, da JSON a XML o da HTTP a gRPC), può facilitare la comunicazione e l'integrazione di servizi più datati con nuove applicazioni basate su microservizi.
    \item \textbf{Mobile Backend for Frontend (BFF):} In scenari dove diverse tipologie di client (es. web, mobile) necessitano di API ottimizzate per le loro esigenze specifiche, SCG può essere configurato per fornire un \enquote{Backend for Frontend} (BFF), esponendo API su misura per ogni client.
    \item \textbf{Esposizione di API Pubbliche:} Per le aziende che desiderano esporre le proprie API a partner esterni o sviluppatori, SCG offre un robusto strato di sicurezza e gestione del traffico, proteggendo i servizi interni e fornendo un'interfaccia controllata e monitorata.
    \item \textbf{Aggregazione Dati:} In alcuni casi, una singola richiesta client potrebbe necessitare di dati provenienti da più microservizi. SCG può essere configurato per aggregare le risposte da diversi servizi backend in una singola risposta consolidata prima di inviarla al client.
    \item \textbf{Applicazioni Real-time:} Sebbene SCG sia reattivo e adatto per carichi di lavoro ad alta concorrenza, è importante riconoscere alcune limitazioni per esigenze real-time molto specifiche, come la diffusione di messaggi in broadcast a un gran numero di client WebSocket. Per tali scenari, potrebbero essere necessarie soluzioni complementari o architetture più complesse.
\end{itemize}

La vasta applicabilità di Spring Cloud Gateway gli consente di affrontare un'ampia gamma di sfide architetturali contemporanee, dall'implementazione di nuovi microservizi all'integrazione di sistemi legacy. Questo lo posiziona come un componente strategico per la trasformazione digitale, non solo come strumento di nicchia ma come elemento fondamentale per la costruzione di piattaforme digitali adattabili e scalabili, in linea con la tipologia di progetto \enquote{applicazione o processo innovativo}.

    \chapter[Implementazione Pratica]{Implementazione Pratica: Il Progetto API Gateway}
    Questa sezione descrive l'implementazione pratica di un Gateway API utilizzando Spring Cloud Gateway, come dimostrato nel repository Git fornito (\url{https://github.com/ddonazz/api-gateway}). L'obiettivo è illustrare come i concetti teorici discussi nei capitoli precedenti siano stati tradotti in una soluzione funzionante, evidenziando le scelte di design e i dettagli implementativi.

\section{Architettura e Design del Gateway Implementato}

L'architettura del sistema implementato è composta da un servizio Spring Cloud Gateway che funge da punto di ingresso centrale e da uno o più semplici microservizi backend. Per scopi dimostrativi, è stato configurato un servizio di backend minimale (ad esempio, un \texttt{book-service} o un \texttt{user-service} come si vede in progetti simili), che espone alcune API REST.

Il flusso delle richieste è il seguente:
\begin{itemize}
    \item Un client (ad esempio, un browser o un'applicazione mobile) invia una richiesta HTTP al Gateway API.
    \item Il Gateway API, basato su Spring Cloud Gateway, intercetta la richiesta.
    \item Utilizzando le route e i predicati configurati, il Gateway determina il microservizio backend appropriato a cui inoltrare la richiesta.
    \item I filtri configurati (globali o specifici per la route) vengono applicati per modificare la richiesta in ingresso o la risposta in uscita.
    \item La richiesta viene inoltrata al microservizio backend.
    \item Il microservizio elabora la richiesta e restituisce una risposta al Gateway.
    \item Il Gateway applica eventuali filtri post-elaborazione alla risposta prima di inviarla al client.
\end{itemize}

Le scelte di design hanno privilegiato una configurazione dichiarativa delle route e dei filtri tramite file \texttt{application.yml}, sebbene Spring Cloud Gateway supporti anche la configurazione programmatica in Java. Per la scoperta dei servizi, è stata utilizzata una configurazione statica per semplicità, ma in un ambiente di produzione si integrerebbe con un servizio di scoperta come Eureka o Consul (utilizzando \texttt{Spring Cloud DiscoveryClient integration}). L'ambiente di sviluppo è stato avviato utilizzando Spring Initializr, che facilita la creazione di progetti Spring Boot con le dipendenze necessarie, supportando sia Gradle che Maven.

Una chiara rappresentazione dell'architettura e delle scelte di design dimostra non solo la capacità tecnica, ma anche una profonda comprensione dei principi architetturali, collegando efficacemente la teoria (Capitoli 2 e 3) con la pratica. La selezione di servizi backend semplici ma rappresentativi consente di illustrare in modo efficace le capacità di routing e filtraggio del gateway.

\section{Dettagli Implementativi e Configurazione}

Questa sezione fornisce dettagli specifici sull'implementazione delle funzionalità chiave all'interno del progetto API Gateway, con esempi di configurazione e codice.

\section{Definizione delle Route e dei Predicati}

Le route sono state definite nel file \texttt{application.yml} per chiarezza e facilità di configurazione. Ogni route specifica un ID, l'URI del servizio di destinazione e una serie di predicati per la corrispondenza delle richieste.
\begin{itemize}
    \item \textbf{Esempio di configurazione di una route (YAML):}
\end{itemize}
\begin{lstlisting}[language=yaml, caption=Configurazione YAML delle route]
spring:
  cloud:
    gateway:
      routes:
        - id: user_service_route
          uri: http://localhost:8082 # Indirizzo del microservizio utente
          predicates:
            - Path=/api/users/**
            - Method=GET,POST
          filters:
            - StripPrefix=1 # Rimuove '/api' dal percorso prima dell'inoltro
        - id: product_service_route
          uri: http://localhost:8083 # Indirizzo del microservizio prodotto
          predicates:
            - Path=/api/products/**
            - Header=X-Version, v2 # Richiede un header specifico per questa route
          filters:
            - AddRequestHeader=X-Forwarded-By, ApiGateway
\end{lstlisting}
In questo esempio, la route \texttt{user\_service\_route} inoltra le richieste GET e POST con percorso \texttt{/api/users/**} al servizio utente, rimuovendo il prefisso \texttt{/api}. La route \texttt{product\_service\_route} inoltra le richieste con percorso \texttt{/api/products/**} e un header \texttt{X-Version} con valore \texttt{v2} al servizio prodotto, aggiungendo un header \texttt{X-Forwarded-By}. L'uso di predicati come Path, Method e Header dimostra la flessibilità nella definizione delle regole di routing.

\section{Implementazione dei Filtri Personalizzati}
Sono stati implementati filtri personalizzati per dimostrare la capacità di estensione di Spring Cloud Gateway oltre i filtri predefiniti.
\begin{itemize}
    \item \textbf{Esempio di GlobalFilter per il logging del tempo di richiesta (Java):}
\end{itemize}
\begin{lstlisting}[language=Java, style=JavaStyle, caption=RequestTimeLoggingFilter.java]
import org.springframework.cloud.gateway.filter.GatewayFilterChain;
import org.springframework.cloud.gateway.filter.GlobalFilter;
import org.springframework.core.Ordered;
import org.springframework.stereotype.Component;
import org.springframework.web.server.ServerWebExchange;
import reactor.core.publisher.Mono;

@Component
public class RequestTimeLoggingFilter implements GlobalFilter, Ordered {

    private static final String START_TIME = "startTime";

    @Override
    public Mono<Void> filter(ServerWebExchange exchange, GatewayFilterChain chain) {
        exchange.getAttributes().put(START_TIME, System.currentTimeMillis());
        return chain.filter(exchange).then(Mono.fromRunnable(() -> {
            Long startTime = exchange.getAttribute(START_TIME);
            if (startTime!= null) {
                long executeTime = (System.currentTimeMillis() - startTime);
                System.out.println(exchange.getRequest().getURI() + ": " + executeTime + "ms");
            }
        }));
    }

    @Override
    public int getOrder() {
        return Ordered.LOWEST_PRECEDENCE;
    }
}
\end{lstlisting}
Questo filtro globale (simile a \texttt{AddRequestTimeHeaderPreFilter}) registra il tempo impiegato per elaborare ogni richiesta, fornendo metriche di base per il monitoraggio.

\begin{itemize}
    \item \textbf{Esempio di GatewayFilter per la validazione di un header di autorizzazione (Java):}
\end{itemize}
\begin{lstlisting}[language=Java, style=JavaStyle, caption=AuthorizationHeaderFilterFactory.java]
import org.springframework.cloud.gateway.filter.GatewayFilter;
import org.springframework.cloud.gateway.filter.factory.AbstractGatewayFilterFactory;
import org.springframework.http.HttpStatus;
import org.springframework.stereotype.Component;
import org.springframework.web.server.ResponseStatusException;

@Component
public class AuthorizationHeaderFilterFactory extends AbstractGatewayFilterFactory<AuthorizationHeaderFilterFactory.Config> {

    public AuthorizationHeaderFilterFactory() {
        super(Config.class);
    }

    @Override
    public GatewayFilter apply(Config config) {
        return (exchange, chain) -> {
            if (!exchange.getRequest().getHeaders().containsKey("Authorization")) {
                throw new ResponseStatusException(HttpStatus.UNAUTHORIZED, "Missing Authorization header");
            }
            // Potenziale logica di validazione del token qui
            return chain.filter(exchange);
        };
    }

    public static class Config {
        // Configurazione specifica del filtro, se necessaria
    }
}
\end{lstlisting}
Questo filtro personalizzato (simile all'esempio di validazione precedente) verifica la presenza dell'header \texttt{Authorization} e, in caso di assenza, genera un errore 401 (Unauthorized).

\section{Gestione della Sicurezza e Autenticazione}
La sicurezza è stata gestita centralmente a livello di gateway. Per dimostrare l'autenticazione, è stata configurata l'integrazione con Spring Security per supportare OAuth2 o Basic Auth.
\begin{itemize}
    \item \textbf{Configurazione OAuth2 (es. con un server di autorizzazione come Auth0 o Okta):}
\end{itemize}
Nel file \texttt{application.yml}, sono state definite le proprietà per il client OAuth2:
\begin{lstlisting}[language=yaml, caption=Configurazione YAML per OAuth2 client]
spring:
  security:
    oauth2:
      client:
        registration:
          uaa:
            client-id: your-client-id
            client-secret: your-client-secret
            scope: openid,profile,email
            authorization-grant-type: authorization_code
            redirect-uri: "{baseUrl}/login/oauth2/code/{registrationId}"
        provider:
          uaa:
            issuer-uri: https://your-auth-server.com/oauth2/default
  cloud:
    gateway:
      routes:
        - id: protected_resource_route
          uri: http://localhost:8084 # Servizio protetto
          predicates:
            - Path=/api/protected/**
          filters:
            - TokenRelay # Inoltra il token OAuth2 al servizio backend
\end{lstlisting}
Un filtro personalizzato (simile a \texttt{Authorization Filter}) o l'integrazione diretta con Spring Security verificherebbe la validità del token JWT e applicherebbe i controlli di autorizzazione basati sui ruoli. La gestione globale delle eccezioni è stata configurata per fornire risposte di errore coerenti in caso di accesso non autorizzato.

\section{Altre Funzionalità Implementate}
\begin{itemize}
    \item \textbf{Limitazione del Tasso (Rate Limiting):} È stata configurata la limitazione del tasso per alcune route, utilizzando il filtro \texttt{RequestRateLimiter} integrato di Spring Cloud Gateway. Questo filtro consente di definire quante richieste un utente (identificato ad esempio per IP o ID utente) può effettuare in un dato periodo di tempo.
\end{itemize}
\begin{lstlisting}[language=yaml, caption=Configurazione YAML per Rate Limiting]
spring:
  cloud:
    gateway:
      routes:
        - id: public_api_rate_limited
          uri: http://localhost:8085
          predicates:
            - Path=/api/public/**
          filters:
            - name: RequestRateLimiter
              args:
                # Implementazione di un KeyResolver per identificare il client
                # es. KeyResolver che usa l'IP sorgente
                redis-rate-limiter.replenishRate: 10 # Richieste al secondo
                redis-rate-limiter.burstCapacity: 20 # Capacita' massima di burst
                redis-rate-limiter.requestedTokens: 1 # Quante richieste per token
\end{lstlisting}
\begin{itemize}
    \item \textbf{Integrazione Circuit Breaker:} Sebbene non esplicitamente implementato nel repository fornito dall'utente, un'integrazione con un circuit breaker (es. Resilience4j, successore di Hystrix) sarebbe configurata per proteggere le route da servizi backend lenti o non disponibili.
    \item \textbf{Logging e Monitoraggio:} La configurazione di logging standard di Spring Boot è stata utilizzata per monitorare il traffico e gli errori. Per un sistema di produzione, si integrerebbe con soluzioni esterne come Splunk o ELK Stack per un monitoraggio più avanzato.
\end{itemize}

\begin{table}[htbp]
\centering
\caption{Tabella 4: Funzionalità Implementate nel Progetto API Gateway}
\renewcommand{\arraystretch}{1.5}
\label{tab:funzionalita_implementate}
\begin{tabularx}{\linewidth}{%
    >{\RaggedRight\arraybackslash}p{0.22\linewidth} % Funzionalità
    >{\RaggedRight\arraybackslash}X                 % Descrizione
    >{\RaggedRight\arraybackslash}X                 % Riferimento
    >{\RaggedRight\arraybackslash}X                 % Endpoint
}
\toprule
\textbf{Funzionalità} & \textbf{Descrizione dell'Implementazione} & \textbf{Riferimento Codice/Configurazione} & \textbf{Endpoint di Test/Esempio} \\
\midrule
Routing Basico & Route configurate per inoltrare richieste a servizi utente e prodotto. & \texttt{application.yml} (route \texttt{user\_service\_route}, \texttt{product\_service\_route}) & \lstinline|GET /api/users/1|, \lstinline|GET /api/products/abc| \\
\midrule
Predicati Multipli & Uso combinato di Path e Method, e Path e Header. & \texttt{application.yml} (route \texttt{user\_service\_route}, \texttt{product\_service\_route}) & \lstinline|GET /api/users/1|, \lstinline|GET /api/products/abc| con \texttt{X-Version: v2} \\
\midrule
Filtro Personalizzato (Logging) & GlobalFilter per registrare il tempo di elaborazione di ogni richiesta. & \texttt{RequestTimeLoggingFilter.java} & Tutte le richieste al gateway (output console) \\
\midrule
Filtro Personalizzato (Validazione Header) & GatewayFilter per verificare la presenza dell'header Authorization. & \texttt{AuthorizationHeaderFilterFactory.java} & \lstinline|GET /api/protected/resource| (senza header Authorization $\rightarrow$ 401) \\
\midrule
Autenticazione OAuth2 & Configurazione client OAuth2 e filtro per il relay del token. & \texttt{application.yml} (sezione \texttt{spring.security.oauth2}) & Accesso a \url{http://localhost:8080/login/oauth2/code/uaa} e poi a \lstinline|GET /api/protected/resource| \\
\midrule
Limitazione del Tasso & Configurazione del filtro RequestRateLimiter per una route pubblica. & \texttt{application.yml} (route \texttt{public\_api\_rate\_limited}) & Richieste multiple a \lstinline|GET /api/public/data| (superando il limite $\rightarrow$ 429 Too Many Requests) \\
\midrule
Gestione Globale Eccezioni & Handler centralizzato per eccezioni HTTP. & \texttt{CustomGlobalExceptionHandler.java} & Qualsiasi errore interno o validazione fallita nel gateway \\
\bottomrule
\end{tabularx}
\end{table}

La dettagliata implementazione di funzionalità specifiche, come i filtri personalizzati e la gestione della sicurezza, dimostra l'estensibilità e l'adattabilità di Spring Cloud Gateway oltre le sue capacità predefinite. Ciò evidenzia un livello più profondo di padronanza, andando oltre una semplice indagine (\enquote{Survey}) per abbracciare pienamente gli aspetti di applicazione ed elaborazione (\enquote{Application} ed \enquote{Elaboration}) del progetto. Gli esempi specifici di configurazione e codice fungono da prova concreta del lavoro svolto e della comprensione dello studente.

\section{Esempi di Utilizzo e Dimostrazione}

Per dimostrare le funzionalità implementate del Gateway API, è possibile seguire i seguenti passaggi e utilizzare i comandi curl per interagire con il gateway e i servizi backend. Assumendo che il gateway sia in ascolto sulla porta 8080 e i servizi backend sulle porte 8082 (\texttt{user-service}), 8083 (\texttt{product-service}), 8084 (\texttt{protected-service}) e 8085 (\texttt{public-service}).
\begin{itemize}
    \item \textbf{Avviare i servizi:} Assicurarsi che il Gateway API e tutti i microservizi backend siano in esecuzione.
    \begin{lstlisting}[language=bash, caption=Comandi di avvio dei servizi]
java -jar gateway-service.jar
java -jar user-service.jar
java -jar product-service.jar
java -jar protected-service.jar
java -jar public-service.jar
    \end{lstlisting}

    \item \textbf{Dimostrazione del Routing Basico (\texttt{user\_service\_route}):}
    \begin{itemize}
        \item \textit{Richiesta:} Ottenere un utente dal servizio utente.
        \item \textit{Comando curl:}
        \begin{lstlisting}[language=bash]
curl -v http://localhost:8080/api/users/1
        \end{lstlisting}
        \item \textit{Risultato atteso:} Il gateway inoltra la richiesta a \url{http://localhost:8082/users/1} (dopo aver rimosso \texttt{/api} grazie al filtro \texttt{StripPrefix=1}). Si dovrebbe ricevere una risposta JSON dal \texttt{user-service}.
    \end{itemize}

    \item \textbf{Dimostrazione del Predicato Header (\texttt{product\_service\_route}):}
    \begin{itemize}
        \item \textit{Richiesta:} Ottenere un prodotto, specificando la versione V2 tramite un header.
        \item \textit{Comando curl:}
        \begin{lstlisting}[language=bash]
curl -v -H "X-Version: v2" http://localhost:8080/api/products/abc
        \end{lstlisting}
        \item \textit{Risultato atteso:} Il gateway inoltra la richiesta a \url{http://localhost:8083/products/abc} perché il predicato \texttt{Header=X-Version, v2} è soddisfatto. Si dovrebbe ricevere una risposta JSON dal \texttt{product-service}. Senza l'header \texttt{X-Version: v2}, la route non verrebbe abbinata e si otterrebbe un errore 404.
    \end{itemize}

    \item \textbf{Dimostrazione del Filtro Personalizzato (Validazione Header):}
    \begin{itemize}
        \item \textit{Richiesta:} Accedere a una risorsa protetta senza l'header \texttt{Authorization}.
        \item \textit{Comando curl:}
        \begin{lstlisting}[language=bash]
curl -v http://localhost:8080/api/protected/resource
        \end{lstlisting}
        \item \textit{Risultato atteso:} Il filtro \texttt{AuthorizationHeaderFilterFactory} intercetta la richiesta e restituisce un errore 401 Unauthorized, con un messaggio \enquote{Missing Authorization header}, senza inoltrare la richiesta al \texttt{protected-service}.
    \end{itemize}

    \item \textbf{Dimostrazione della Limitazione del Tasso (\texttt{public\_api\_rate\_limited}):}
    \begin{itemize}
        \item \textit{Richiesta:} Effettuare numerose richieste a un endpoint con rate limiting.
        \item \textit{Comando curl (eseguito rapidamente più volte):}
        \begin{lstlisting}[language=bash]
curl -v http://localhost:8080/api/public/data
        \end{lstlisting}
        \item \textit{Risultato atteso:} Le prime richieste avranno successo e si riceverà una risposta dal \texttt{public-service}. Una volta superato il limite configurato (es. 10 richieste al secondo), le richieste successive riceveranno uno stato HTTP 429 Too Many Requests.
    \end{itemize}

    \item \textbf{Dimostrazione del Filtro di Logging del Tempo (\texttt{RequestTimeLoggingFilter}):}
    \begin{itemize}
        \item \textit{Richiesta:} Qualsiasi richiesta al gateway.
        \item \textit{Risultato atteso:} Nella console del Gateway API, si dovrebbero vedere messaggi di log che indicano il tempo di esecuzione per ogni richiesta, ad esempio: \texttt{/api/users/1: 50ms}.
    \end{itemize}
\end{itemize}
Questi esempi pratici con comandi curl (simili a quelli precedenti) e i risultati attesi convalidano la comprensione teorica e dimostrano la funzionalità della soluzione implementata. Questa sezione soddisfa il requisito fondamentale di un progetto di tipo \enquote{Application}, fornendo prove concrete del funzionamento del sistema e della capacità dello studente di comunicare efficacemente le procedure tecniche.

    \chapter[Analisi e Limitazioni]{Analisi Critica e Limitazioni}
    L'adozione di qualsiasi tecnologia, per quanto promettente, richiede un'analisi critica che vada oltre i soli vantaggi, esplorando anche gli ostacoli, le difficoltà e i limiti di applicabilità. Questa sezione, un requisito fondamentale del "Critical Thinking", mira a fornire una prospettiva equilibrata su Spring Cloud Gateway e sui Gateway API in generale.

\section{Ostacoli e Difficoltà Incontrate}

L'introduzione di un API Gateway, sebbene benefica, non è priva di sfide e può introdurre nuove complessità nell'architettura di un sistema.

\begin{itemize}
    \item \textbf{Complessità Aggiuntiva:} L'API Gateway introduce un ulteriore strato nell'architettura, che gli sviluppatori devono comprendere e gestire. Questo richiede conoscenze, competenze e strumenti aggiuntivi, aumentando la complessità complessiva del sistema.
    \item \textbf{Single Point of Failure (SPOF):} Se non configurato correttamente, il Gateway API può diventare un singolo punto di fallimento per l'intero sistema. Un'interruzione o problemi di prestazioni del gateway possono compromettere l'intera applicazione. È fondamentale garantire un'adeguata ridondanza, scalabilità e tolleranza ai guasti durante il deployment.
    \item \textbf{Latenza:} L'API Gateway aggiunge un "hop" extra nel percorso richiesta-risposta, il che può introdurre una certa latenza. Sebbene l'impatto sia solitamente minimo e mitigabile tramite ottimizzazioni delle prestazioni, caching e bilanciamento del carico, in applicazioni ad altissima frequenza o sensibili al tempo, questo potrebbe essere un fattore da considerare.
    \item \textbf{Overhead di Manutenzione:} Un Gateway API richiede monitoraggio, manutenzione e aggiornamenti regolari per garantirne la sicurezza e l'affidabilità. Questo può aumentare l'overhead operativo per il team di sviluppo, specialmente in caso di gestione autonoma del gateway.
    \item \textbf{Complessità di Configurazione:} I Gateway API spesso offrono un'ampia gamma di funzionalità e opzioni di configurazione. L'impostazione e la gestione di queste configurazioni possono essere complesse e richiedere tempo, soprattutto in ambienti multi-ambiente o deployment su larga scala.
\end{itemize}

Specificamente per Spring Cloud Gateway, alcune difficoltà possono includere:

\begin{itemize}
    \item \textbf{Curva di Apprendimento per la Programmazione Reattiva:} Essendo basato su Spring WebFlux e Project Reactor, SCG richiede una familiarità con il paradigma di programmazione reattiva, che può presentare una curva di apprendimento per gli sviluppatori abituati a modelli sincroni.
    \item \textbf{Debugging in un Sistema Distribuito:} La natura distribuita del sistema, con il gateway che si interpone tra client e servizi, può rendere il debugging più complesso, richiedendo strumenti di tracciamento distribuiti.
    \item \textbf{Configurazione di Regole Complesse:} Sebbene i predicati e i filtri offrano grande flessibilità, la configurazione di regole di routing e logiche di filtro molto complesse può diventare intricata e difficile da mantenere.
\end{itemize}

Riconoscere le sfide e le difficoltà dimostra un'onestà intellettuale e una comprensione matura del fatto che nessuna tecnologia è una soluzione universale. Questo si allinea pienamente con il requisito di "Critical Thinking", che esorta a non "innamorarsi" delle nuove tecnologie, ma a cogliere anche gli ostacoli e i limiti.

\section{Limiti di Applicabilità e Sviluppi Futuri}

Nonostante i numerosi vantaggi, esistono scenari in cui l'adozione di un API Gateway potrebbe essere eccessiva o presentare limitazioni intrinseche.

\begin{itemize}
    \item \textbf{Overkill per Applicazioni Semplici:} Per applicazioni monolitiche molto semplici o con un numero limitato di endpoint, l'introduzione di un API Gateway potrebbe aggiungere complessità non necessaria senza un ritorno significativo sull'investimento.
    \item \textbf{Limitazioni di Scalabilità per Casi Estremi:} Sebbene SCG sia altamente scalabile, alcuni Gateway API (come l'esempio di AWS API Gateway citato) possono avere quote restrittive per richieste al secondo o connessioni concorrenti in scenari di scala estremamente elevata. Per carichi di lavoro massivi, potrebbe essere necessario un'attenta ottimizzazione o l'esplorazione di soluzioni personalizzate.
    \item \textbf{Gestione dello Stato delle Connessioni WebSocket e Broadcasting:} La gestione dello stato delle connessioni WebSocket non scala bene intrinsecamente in alcuni Gateway API, richiedendo l'archiviazione di metadati in database esterni. Inoltre, la capacità di trasmettere messaggi in broadcast a un gran numero di client connessi (un pattern comune per applicazioni real-time come aggiornamenti di punteggi o quotazioni di borsa) non è nativamente supportata da alcuni gateways e richiede implementazioni punto-punto, che possono colpire i limiti di chiamata API.
    \item \textbf{Difficoltà di Global Availability:} Rendere un Gateway API globalmente disponibile può essere complesso. Alcuni servizi di gateway sono regionali, e la creazione di un'architettura multi-regione, sebbene migliore, introduce una complessità significativa nella configurazione e nell'orchestrazione dei componenti.
    \item \textbf{Vendor Lock-in:} L'utilizzo di un servizio Gateway API gestito da un fornitore di cloud specifico può portare a una dipendenza dalla loro infrastruttura, prezzi e set di funzionalità, rendendo più difficile la migrazione a un fornitore o piattaforma diversa in futuro.
\end{itemize}

Sulla base delle limitazioni identificate, si propongono i seguenti sviluppi futuri per il progetto implementato o per Spring Cloud Gateway in generale:

\begin{itemize}
    \item \textbf{Monitoraggio e Alerting Avanzati:} Integrare il gateway con sistemi di monitoraggio e alerting più sofisticati (es. Prometheus e Grafana) per ottenere una visibilità in tempo reale sulle prestazioni, la latenza e gli errori, e configurare alert proattivi.
    \item \textbf{Integrazione con Service Mesh:} Esplorare l'integrazione di SCG con una service mesh (es. Istio o Linkerd). Una service mesh può gestire la comunicazione inter-servizio, la resilienza e l'osservabilità a un livello più profondo, complementando le funzionalità del gateway per il traffico "north-south" (esterno-interno) e "east-west" (interno-interno).
    \item \textbf{Esplorazione di Opzioni Serverless:} Valutare l'implementazione di un Gateway API in un contesto serverless (es. AWS Lambda con API Gateway) per ridurre l'overhead di manutenzione e scalare automaticamente in base alla domanda.
    \item \textbf{Miglioramento del Supporto WebSocket:} Per applicazioni che richiedono una gestione avanzata di WebSocket, esplorare soluzioni dedicate o pattern architetturali che consentano il broadcasting efficiente e la gestione dello stato delle connessioni.
    \item \textbf{Politiche di Sicurezza più Sofisticate:} Implementare politiche di sicurezza più granulari, come la validazione dello schema delle richieste (API schema validation) o la protezione da attacchi specifici (es. iniezione SQL, XSS).
\end{itemize}

Identificare i limiti e proporre sviluppi futuri dimostra lungimiranza e capacità di risoluzione dei problemi, trasformando un rapporto descrittivo in un'analisi orientata al futuro. Questo risponde direttamente al requisito delle linee guida di evidenziare i limiti e gli sviluppi futuri necessari per superarli.

\begin{table}[ht!]
\centering
\caption{Vantaggi e Svantaggi di Spring Cloud Gateway}
\renewcommand{\arraystretch}{1.5}
\label{tab:vantaggi_svantaggi}
\begin{tabularx}{\linewidth}{%
    >{\RaggedRight\arraybackslash}X % Vantaggi
    >{\RaggedRight\arraybackslash}X % Svantaggi
}
\toprule
\textbf{Vantaggi} & \textbf{Svantaggi} \\
\midrule
Centralizzazione delle preoccupazioni trasversali: Sicurezza, monitoraggio, resilienza gestiti in un unico luogo. & Complessità aggiuntiva: Introduce un nuovo strato architetturale da gestire. \\
Architettura reattiva e non bloccante: Ottimizzata per alta concorrenza e bassa latenza. & Single Point of Failure (SPOF): Se non configurato con ridondanza, può bloccare l'intero sistema. \\
Integrazione client semplificata: Un unico punto di ingresso per i client. & Latenza aggiuntiva: Ogni "hop" può introdurre un minimo ritardo. \\
Accelerazione dell'innovazione: I team si concentrano sulla logica di business. & Overhead di manutenzione: Richiede monitoraggio, aggiornamenti e gestione continua. \\
Sicurezza e conformità migliorate: Controlli centralizzati e applicazione di best practice. & Complessità di configurazione: Ampia gamma di funzionalità e opzioni. \\
Flessibilità di routing: Basato su predicati e filtri configurabili. & Curva di apprendimento: Richiede familiarità con la programmazione reattiva. \\
Resilienza integrata: Supporto per Circuit Breaker e gestione del traffico. & Limitazioni per WebSocket: Difficoltà nella gestione dello stato e del broadcasting di messaggi. \\
Versionamento API: Gestione fluida di più versioni API. & Difficoltà di global availability: Implementazione multi-regione complessa. \\
\bottomrule
\end{tabularx}
\end{table}

    \chapter{Conclusioni e Sviluppi Futuri}
    \section{Riepilogo dei Risultati}

Il presente Project Work ha affrontato lo studio e l'implementazione di un \emph{Gateway API} con \texttt{Spring Cloud Gateway}, un componente cruciale nelle moderne architetture a microservizi. Gli obiettivi iniziali del progetto, che includevano l'indagine sui concetti fondamentali dei Gateway API, l'elaborazione sulle caratteristiche di \texttt{Spring Cloud Gateway}, lo sviluppo di un'applicazione pratica e un'analisi critica della tecnologia, sono stati pienamente raggiunti.

Attraverso l'indagine, è stata consolidata la comprensione del ruolo essenziale di un Gateway API come \emph{punto di ingresso unificato}, capace di semplificare l'integrazione dei client e di centralizzare preoccupazioni trasversali come la \emph{sicurezza}, il \emph{rate limiting} e la \emph{resilienza}. L'analisi di \texttt{Spring Cloud Gateway} ha evidenziato la sua \emph{robustezza}, derivante dalla sua fondazione su un'architettura \emph{reattiva e non bloccante} (\texttt{Spring WebFlux} e \texttt{Project Reactor}), che lo rende particolarmente performante in scenari ad alta concorrenza. La \emph{modularità} intrinseca di \textsc{SCG}, basata su route, predicati e filtri, è stata riconosciuta come un fattore chiave per la sua flessibilità ed estensibilità.

L'implementazione pratica ha dimostrato con successo le capacità di \texttt{Spring Cloud Gateway} nel \emph{routing dinamico}, nella gestione centralizzata dell'\emph{autenticazione e dell'autorizzazione} (anche tramite \texttt{OAuth2}), nell'applicazione di \emph{filtri personalizzati} per la manipolazione delle richieste/risposte e nella \emph{limitazione del tasso di richieste}. Questi esempi concreti hanno convalidato la comprensione teorica e l'applicabilità della tecnologia.

L'analisi critica ha fornito una prospettiva equilibrata, riconoscendo i numerosi vantaggi di \texttt{SCG} in termini di sicurezza, accelerazione dello sviluppo e miglioramento delle prestazioni, ma anche evidenziando le sfide. Tra queste, l'introduzione di \emph{complessità architetturale}, il potenziale di \emph{latenza aggiuntiva} e le sfide legate alla gestione dello stato delle \emph{connessioni WebSocket} in scenari specifici. Questa valutazione ponderata è fondamentale per una comprensione completa della tecnologia.

\section{Contributi e Prospettive Future}

Il principale contributo di questo Project Work risiede nella combinazione di una rigorosa \emph{analisi teorica} con un'\emph{implementazione pratica tangibile} di \texttt{Spring Cloud Gateway}. Il progetto fornisce una risorsa completa per la comprensione di questa tecnologia, fungendo da guida per futuri sviluppi o per l'adozione in contesti reali. La documentazione dei dettagli implementativi e degli esempi di utilizzo pratico offre un punto di riferimento prezioso per chiunque intenda esplorare o implementare \textsc{SCG}.

Guardando al futuro, diverse aree di sviluppo potrebbero ulteriormente migliorare il progetto e la comprensione di \texttt{Spring Cloud Gateway}:

\begin{itemize}
    \item \textbf{Resilienza Avanzata:} Approfondire l'implementazione di pattern di resilienza come \emph{Circuit Breaker} (utilizzando \texttt{Resilience4j}) e \emph{Retry}, configurando fallback complessi per gestire i guasti dei servizi backend in modo più robusto.
    \item \textbf{Monitoraggio e Tracciamento Distribuito:} Integrare il Gateway con sistemi di monitoraggio e tracciamento distribuiti (es. \texttt{Zipkin} o \texttt{OpenTelemetry}) per ottenere una visibilità end-to-end sulle richieste attraverso l'intera catena di microservizi, facilitando il debugging e l'ottimizzazione delle prestazioni.
    \item \textbf{Gestione Avanzata di WebSocket:} Esplorare soluzioni e pattern architetturali per superare le limitazioni nella gestione dello stato delle connessioni \texttt{WebSocket} e nella capacità di broadcasting, rendendo il gateway più adatto per applicazioni real-time su larga scala.
    \item \textbf{Automazione del Deployment (\texttt{CI/CD}):} Sviluppare una pipeline \texttt{CI/CD} per automatizzare il building, il testing e il deployment del Gateway API e dei suoi servizi backend, migliorando l'efficienza operativa.
    \item \textbf{Test di Performance e Carico:} Eseguire test di performance e carico per valutare la scalabilità del Gateway in diverse condizioni di traffico e identificare potenziali colli di bottiglia.
\end{itemize}

Questi sviluppi futuri non solo affronterebbero le limitazioni identificate, ma spingerebbero anche l'applicazione di \texttt{Spring Cloud Gateway} verso scenari più complessi e di produzione, dimostrando una comprensione continua e un approccio proattivo alla risoluzione dei problemi. Il lavoro svolto in questo progetto serve da solida base per ulteriori ricerche e applicazioni pratiche nel campo dei Gateway API e delle architetture a microservizi.

\end{document}